\documentclass[a4paper, oneside, 12pt, openany]{book}
\pdfoutput=1

\usepackage{packages}
\usepackage{macros}

% Label tables just like equations, theorems, definitions, etc.
%
% NB: This can be confusing if LaTeX does not place the table at the point of
% writing (e.g. for lack of space)!
\numberwithin{equation}{chapter}
\numberwithin{table}{chapter}
\makeatletter
\let\c@equation\c@table
\makeatother

% Setting up the coloured environments
%
\newbool{shade-envs}
% This can be used to toggle the coloured environments on or off.
\setboolean{shade-envs}{true}

%%
\ifthenelse{\boolean{shade-envs}}{%
  % Colours are as in Andrej Bauer's notes on realizability:
  % https://github.com/andrejbauer/notes-on-realizability
  \colorlet{ShadeOfPurple}{blue!5!white}
  \colorlet{ShadeOfYellow}{yellow!5!white}
  \colorlet{ShadeOfGreen} {green!5!white}
  \colorlet{ShadeOfBrown} {brown!10!white}
  % But we also shade proofs
  \colorlet{ShadeOfGray}  {gray!10!white}
  % For exercises
  \colorlet{ShadeOfRed}   {red!10!white}
}
% If we don't want to have shaded environments, then we use a closing symbol
% \lozenge to mark the end of remarks, definitions and examples.
{%
  \declaretheoremstyle[
      spaceabove=6pt,
      spacebelow=6pt,
      bodyfont=\normalfont,
      qed=\(\lozenge\)
  ]{definitionwithbox}
  \declaretheoremstyle[
      headfont=\itshape,
      bodyfont=\normalfont,
      qed=\(\lozenge\)
      ]{remarkwithbox}
}

% Now we set the shading using the tcolorbox package.
%
% The related thmtools' option "shaded" and the package mdframed seem to have
% issues: the former does not allow for page breaks in shaded environments and
% the latter puts double spacing between two shaded environments.
\tcbset{shadedenv/.style={
    colback={#1},
    frame hidden,
    enhanced,
    breakable,
    boxsep=0pt,
    left=2mm,
    right=2mm,
    % LaTeX thinks this is too wide (as becomes clear from the many "Overfull
    % \hbox" warnings, but optically it looks spot on.
    add to width=1.1mm,
    enlarge left by=-0.6mm}
}

\ifthenelse{\boolean{shade-envs}}{%
  \declaretheorem[sibling=equation]{theorem}
  \declaretheorem[unnumbered,title=Theorem]{theorem*}
  \declaretheorem[sibling=theorem]{lemma,proposition,corollary}
  \declaretheorem[unnumbered,title=Lemma]{lemma*}
  \declaretheorem[sibling=theorem,style=definition]{definition}
  \declaretheorem[sibling=theorem,style=definition]{example}
  \declaretheorem[sibling=theorem,style=remark]{remark}
  \declaretheorem[sibling=theorem,style=definition]{exercise}
  %
  \tcolorboxenvironment{theorem}    {shadedenv={ShadeOfPurple}}
  \tcolorboxenvironment{theorem*}   {shadedenv={ShadeOfPurple}}
  \tcolorboxenvironment{lemma}      {shadedenv={ShadeOfPurple}}
  \tcolorboxenvironment{lemma*}     {shadedenv={ShadeOfPurple}}
  \tcolorboxenvironment{proposition}{shadedenv={ShadeOfPurple}}
  \tcolorboxenvironment{corollary}  {shadedenv={ShadeOfPurple}}
  \tcolorboxenvironment{definition} {shadedenv={ShadeOfYellow}}
  \tcolorboxenvironment{example}    {shadedenv={ShadeOfGreen}}
  \tcolorboxenvironment{remark}     {shadedenv={ShadeOfBrown}}
  \tcolorboxenvironment{proof}      {shadedenv={ShadeOfGray}}
  \tcolorboxenvironment{exercise}   {shadedenv={ShadeOfRed}}
}{% Use closing symbols if we don't have colours
  \declaretheorem[sibling=equation]{theorem}
  \declaretheorem[sibling=theorem]{lemma,proposition,corollary}
  \declaretheorem[unnumbered,title=Theorem]{theorem*}
  \declaretheorem[unnumbered,title=Lemma]{lemma*}
  \declaretheorem[sibling=theorem,style=definitionwithbox]{definition}
  \declaretheorem[sibling=theorem,style=definitionwithbox]{example}
  \declaretheorem[sibling=theorem,style=remarkwithbox]{remark}
  \declaretheorem[sibling=theorem,style=definitionwithbox]{exercise}
  \tcolorboxenvironment{theorem}    {shadedenv={white}}
  \tcolorboxenvironment{theorem*}   {shadedenv={white}}
  \tcolorboxenvironment{lemma}      {shadedenv={white}}
  \tcolorboxenvironment{lemma*}     {shadedenv={white}}
  \tcolorboxenvironment{proposition}{shadedenv={white}}
  \tcolorboxenvironment{corollary}  {shadedenv={white}}
  \tcolorboxenvironment{definition} {shadedenv={white}}
  \tcolorboxenvironment{example}    {shadedenv={white}}
  \tcolorboxenvironment{remark}     {shadedenv={white}}
  \tcolorboxenvironment{proof}      {shadedenv={white}}
  \tcolorboxenvironment{exercise}   {shadedenv={white}}
  }
  \declaretheorem[sibling=theorem,style=remark,numbered=no]{claim}

% Note that proofs will still have the \qed symbol at the end, even when shaded,
% because we prefer to keep up the tradition.


\includeonly{mainmatter/introduction}

\begin{document}

\frontmatter

\begin{titlepage}
\begin{center}
  \vspace*{\stretch{0.5}}

  \large % Default size for the title page

  {\Huge\textsc{Domain theory and \\ denotational semantics}\par}

  \vspace{\stretch{0.2}}

  by

  \vspace{\stretch{0.2}}

  {\huge\textsc{Tom de Jong}}

  \vspace{\stretch{0.5}}

  {\Large{Lecture notes and exercises for the\\
      \textsc{Midlands Graduate School (MGS)}}} \\
  \vspace{\stretch{0.1}}
  2--6 April 2023, Birmingham, UK

  \vspace{\stretch{1}}

  \begingroup
  \tikzset{every picture/.style={color=Gray!90!Black}}
  \begin{tikzcd}
    0 & 1 & 2 & 3 & \ldots \\
    & & \bot \ar[ull,no head] \ar[ul,no head] \ar[u,no head] \ar[ur, no head] \ar[urr, no head]
  \end{tikzcd}
  \endgroup

  \vfill

  \flushright
  {\normalsize{School of Computer Science \\
  University of Nottingham \\
  February--March 2023}}

\end{center}
\end{titlepage}
\restoregeometry%

\chapter{Abstract}

Denotational semantics aims to understand computer programmes by assigning
mathematical meaning to the syntax of a programming language.
%
In this course we will study a simple functional programming language called
PCF. Notably, this language has general recursion through a fixed point
operator.
%
This means a simple denotational semantics based on sets is not
suitable. Instead, we interpret the types of PCF as certain partially ordered
sets leading to domain theory and Scott's model of PCF in particular.
%
The central theorems of soundness and computational adequacy, formulated and
proved by Plotkin, then tell us that a PCF programme computes to a value if and
only if their interpretations in the model are equal.


%%% Local Variables:
%%% mode: latexmk
%%% TeX-master: "../main"
%%% End:

\chapter{Acknowledgements}

It is my pleasure to thank Mart\'in Escard\'o who introduced me to the beautiful
subjects of denotational semantics and domain theory. I am also grateful for his
comments on earlier drafts of these notes.

Additionally, I wish to express my thanks to Ayberk Tosun for proof-reading and
suggesting several improvements.


%%% Local Variables:
%%% mode: latexmk
%%% TeX-master: "../main"
%%% End:


% \setcounter{tocdepth}{2}
% \tableofcontents

\mainmatter%

\chapter{Introduction}

\section{Aims}

\section{References}

\cite{AbramskyJung1994,GierzEtAl2003,Streicher2006,Hart2020}

\section{Acknowledgements}

It is my pleasure to thank Mart\'in Escard\'o who introduced me to the beautiful
subjects of denotational semantics and domain theory. I am also grateful for his
comments on earlier drafts of these notes.


%%% Local Variables:
%%% mode: latexmk
%%% TeX-master: "../main"
%%% End:

\chapter{PCF and its operational semantics}\label{chap:PCF}

PCF (Programming Computable Functions)~\cite{Plotkin1977} is a typed functional
programming language with general recursion. We can think of PCF as a simpler,
well-behaved fragment of a modern functional programming language such as
Haskell~\cite{Haskell2010}.
%
The point of PCF is not to be a particularly convenient or rich programming
language. Instead, it's meant to be a simple and principled language enabling us
to study it from a mathematical viewpoint without having to deal with complex
features that you might find in modern, real-world programming languages.
%
The techniques that we will employ could, with some effort, be extended to more
complex programming languages however, see e.g.~\cite{Plotkin1983}.

At the same time, it should be mentioned that PCF is not quite a toy language
and has a rather rich theory. For example, it captures Kleene--Kreisel
higher-type computability~\cite{LongleyNormann2015} and an extension of
PCF---with parallel-or and \(\exists\) (see~\cite{Streicher2006} for
details)---can simulate recursively defined datatypes (such as trees and lists)
using retracts~\cite{Streicher1994}.

\section{PCF}

We describe the syntax of PCF and its small-step operational semantics which
describe how to compute in PCF.

\begin{definition}[Types of PCF, \(\pcfnat\), \(\sigma \pcffun \tau\)]
  The \emph{types of PCF} are inductively defined as:
  \begin{enumerate}[(i)]
  \item \(\pcfnat\) is the \emph{base type} of PCF, and
  \item if \(\sigma\) and \(\tau\) are types of PCF, then we have the
    \emph{function type} \(\sigma \pcffun \tau\).
  \end{enumerate}
  Moreover, as usual, we will write \(\sigma \pcffun \tau \pcffun \rho\) for
  \(\sigma \pcffun (\tau \pcffun \rho)\).
\end{definition}

From now on, such inductive definitions will be presented in the following style:
\begin{center}
  \AxiomC{\phantom{\(\pcfnat\)}}
  \UnaryInfC{\(\pcfnat\) is a type of PCF}
  \DisplayProof\hspace{3cm}
  \AxiomC{\(\sigma\) is a type of PCF}
  \AxiomC{\(\tau\) is a type of PCF}
  \BinaryInfC{\(\pa*{\sigma \pcffun \tau}\) is a type of PCF}
  \DisplayProof
\end{center}

For each type \(\sigma\), we assume to have a countably infinite set of typed
variables, typically denoted by \(\var x : \sigma\), \(\var y : \sigma\), or
\(\var x_1 : \sigma\), \(\var x_2 : \sigma\), etc.

\begin{definition}[Context, \(\Gamma\)]\label{def:context}
  A \emph{context} \(\Gamma\) is a list of variables:
  \[
    \Gamma = [\var x_0 : \sigma_0 , \var x_1 : \sigma_1 , \dots , \var x_{n-1} :
    \sigma_{n-1}].
  \]
  For \(n = 0\), we get the \emph{empty context}: an empty list with no
  variables.
\end{definition}

We are now ready to define the (well-typed) terms of PCF.

\begin{definition}[Terms of PCF, \(\Gamma \vdash M : \sigma\)]\label{def:PCF-terms}
  We inductively define the \emph{terms \(M\) of type~\(\sigma\) in
    context~\(\Gamma\)}, written \(\Gamma \vdash M : \sigma\), by the following
  clauses:
  \begin{center}
  \def\fCenter{\ \vdash\ }

  \AxiomC{\phantom{$\fCenter$}}
  \UnaryInf$\Gamma,\var{x}:\sigma,\Delta \fCenter \var{x} : \sigma$
  \DisplayProof\hspace{3.7cm}
  \Axiom$\Gamma , \var{x} : \sigma \fCenter M : \tau$
  \UnaryInf$\Gamma \fCenter (\lambdadot{\var{x} : \sigma}{M}) : \sigma \pcffun \tau$
  \DisplayProof\vspace{1cm}\\
  \Axiom$\Gamma \fCenter M : \sigma \pcffun \tau$
  \Axiom$\Gamma \fCenter N : \sigma$
  \BinaryInf$\Gamma \fCenter M(N) : \tau$
  \DisplayProof\hspace{3.9cm}
  \Axiom$\Gamma \fCenter M : \sigma \pcffun \sigma$
  \UnaryInf$\Gamma \fCenter \pcffix_\sigma(M) : \sigma$
  \DisplayProof\vspace{1cm}\\
  \AxiomC{}
  \UnaryInf$\Gamma \fCenter \pcfzero : \pcfnat$
  \DisplayProof\quad\quad\quad
  \Axiom$\Gamma \fCenter M : \pcfnat$
  \UnaryInf$\Gamma \fCenter \pcfsuc(M) : \pcfnat$
  \DisplayProof\quad\quad\quad
  \Axiom$\Gamma \fCenter M : \pcfnat$
  \UnaryInf$\Gamma \fCenter \pcfpred(M) : \pcfnat$
  \DisplayProof\vspace{1cm}\\
  \Axiom$\Gamma \fCenter M : \pcfnat$
  \Axiom$\Gamma \fCenter N_1 : \pcfnat$
  \Axiom$\Gamma \fCenter N_2 : \pcfnat$
  \TrinaryInf$\Gamma \fCenter \pcfifz(M,N_1,N_2) : \pcfnat$
  \DisplayProof
\end{center}
When \(\Gamma\) is the empty context, we simply write \(\vdash M : \sigma\) and
we call \(M\) a \emph{closed} term or a \emph{program}.
%
Note that programs do not contain any free variables.

We will often write \(M \, N\) instead of \(M(N)\) to ease readability.
\end{definition}

The first three rules above will look familiar to someone who has seen the typed
\(\lambda\)-calculus before and basically say that we form functions that we can
apply. The three rules on the third row give us natural numbers with a
predecessor constructor.
%
We illustrate the remaining rules, those for \(\pcffix\) and \(\pcfifz\),
in~\cref{exam:ifzero,exam:addition-by-n} below.

So far, we only have some terms, but we have not specified any computational
behaviour of those terms yet. \cref{def:small-step} defines a reduction strategy
that specifies the computational behaviour of PCF and should help us understand
the intended meaning of the terms.

\begin{definition}[Numeral, \(\numeral{n}\)]
  For a natural number \(n \in \Nat\), we define the
  \emph{numeral}~\(\numeral n\) in PCF inductively:
  \(\numeral 0 \coloneqq \pcfzero\) and
  \(\numeral{m+1} \coloneqq \pcfsuc \, \numeral m\).
\end{definition}

\begin{definition}[Small-step operational semantics of PCF, \(M \smallstep N\)]%
  \label{def:small-step}%
  We inductively define when a term \(M\) \emph{(small-step) reduces} to another
  term \(N\) (of the same type, in the same context), written
  \(M \smallstep N\), by the following inductive clauses:
  %\begin{center}
    \begin{longtable}{cc}
      \AxiomC{\phantom{${\smallstep}$}}
      \UnaryInfC{\((\lambdadot{\var x : \sigma}{M})N \smallstep M[N/\var x]\)}
      \DisplayProof
      &
        \AxiomC{\phantom{${\smallstep}$}}
        \UnaryInfC{\(\pcffix_\sigma \, M \smallstep M\pa*{\pcffix_\sigma \, M}\)}
        \DisplayProof\vspace{.5cm}\\
      \AxiomC{\phantom{${\smallstep}$}}
      \UnaryInfC{\(\pcfpred \, \numeral 0 \smallstep \numeral 0\)}
      \DisplayProof
      &
      \AxiomC{\phantom{${\smallstep}$}}
      \UnaryInfC{\(\pcfpred \, \numeral{n+1} \smallstep \numeral n\)}
      \DisplayProof\vspace{.5cm}\\
      \AxiomC{\phantom{${\smallstep}$}}
      \UnaryInfC{\(\pcfifz(\numeral 0,M,N) \smallstep M\)}
      \DisplayProof
      &
      \AxiomC{\phantom{${\smallstep}$}}
      \UnaryInfC{\(\pcfifz(\numeral {n+1},M,N) \smallstep N\)}
      \DisplayProof\vspace{.5cm}\\
      \AxiomC{\(M_1 \smallstep M_2\)}
      \UnaryInfC{\(M_1 \, N \smallstep M_2 \, N\)}
      \DisplayProof
      &
      \AxiomC{\(M_1 \smallstep M_2\)}
      \UnaryInfC{\(\pcfsuc \, M_1 \smallstep \pcfsuc \, M_2\)}
      \DisplayProof\vspace{.5cm}\\
      \AxiomC{\(M_1 \smallstep M_2\)}
      \UnaryInfC{\(\pcfpred \, M_1  \smallstep \pcfpred \, M_2\)}
      \DisplayProof
      &
      \AxiomC{\(M_1 \smallstep M_2\)}
      \UnaryInfC{\(\pcfifz(M_1,N_1,N_2) \smallstep \pcfifz(M_2,N_1,N_2)\)}
      \DisplayProof
    \end{longtable}
  %\end{center}
  Here \(M[N/x]\) denotes the result of substituting \(N\) for the variable
  \(\var x\) in \(M\).
\end{definition}

The final three rules are like congruence rules saying that you can reduce a
term \(\pcfsuc M\) by reducing the inner term \(M\).
%
The reduction \(\pcffix_{\sigma}\,M \smallstep M(\pcffix_\sigma \, M)\)
corresponds to the idea that \(\pcffix_\sigma\,M\) is a \emph{fixed point} of
\(M : \sigma \pcffun \sigma\). Alternatively, we can think of it as a single
unfolding of a recursive definition, as illustrated
in~\cref{exer:small-step-addition,exam:non-termination}.

\begin{example}\label{exam:ifzero}
  We can translate the program \verb|if (x + 1 == 0) then 5 else 3|, written in
  pseudocode, to the PCF program
  \(\lambdadot{\var x : \pcfnat}{\pcfifz(\pcfsuc \, \var x, \numeral 5, \numeral 3)}\).
\end{example}



The point of \(\pcffix\) is that it gives us general recursion, as we will
explain with an example now.

\begin{example}[Addition by \(n\) in PCF]\label{exam:addition-by-n}
  For a natural number \(n \in \Nat\), consider addition by \(n\) as a
  recursively defined function:
  \begin{alignat*}{3}
    &\add_n : \Nat \to && \Nat && \\
    &\add_n(0) &&\coloneqq n, && \\
    &\add_n(k+1) &&\coloneqq \add_n(k) + 1. &&
  \end{alignat*}
  We show how to write this function as a PCF program. We start by slightly
  rewriting \(\add_n\) as:
  \begin{equation*}\label{add_n-alt}\tag{\(\ast\)}
    \add_n(m) \coloneqq
    \begin{cases}
      n &\text{if } m = 0, \\
      \suc\pa*{\add_n(\pred(m))} &\text{else}.
    \end{cases}
  \end{equation*}
  Looking at the above, we see that we have most of the analogues readily
  available in PCF: \(\pcfifz\), \(\pcfsuc\) and \(\pcfpred\), as well as the
  numeral \(\underline n\).

  We next define the program
  \(F : \pa*{\pcfnat \pcffun \pcfnat} \pcffun \pa*{\pcfnat \pcffun \pcfnat}\) as
  follows:
  \[
    F \coloneqq \lambdadot{\var f : \pa{\pcfnat \pcffun
        \pcfnat}}{\lambdadot{\var y : \pcfnat}%
      {\pcfifz\pa*{\var y,\numeral{n},%
          \pcfsuc\pa*{\var f\pa*{\pcfpred \, \var y }}}}}.
  \]
  In the program \(F\), the variable \(\var y\) plays the role of \(m\) in
  \eqref{add_n-alt} and \(\var x\) is like a placeholder for the recursive call.

  Mirroring~\eqref{add_n-alt} further, what we want is a program
  \(f : \pcfnat \pcffun \pcfnat\) such that \(f\) is ``equal'' to \(F\,f\). That
  is, we want a \emph{fixed point} of \(F\).
  %
  Hence, we finally define \(\pcfadd_n\) as:
  \[
    \pcfadd_n \coloneqq \pcffix_{\pcfnat \pcffun \pcfnat} \, F. \qedhere
  \]
  % \[
  %   \pcffix_{\pcfnat \pcffun \pcfnat} : \pa*{\pa*{\pcfnat \pcffun \pcfnat}
  %     \pcffun \pa*{\pcfnat \pcffun \pcfnat}} \pcffun \pa{\pcfnat \pcffun \pcfnat}.
  % \]
\end{example}

\begin{exercise}\label{exer:small-step-addition}
  Give sequences of small-step reductions showing that
  \(\pcfadd_n \, \numeral 0\) and \(\pcfadd_n \, \numeral 1\) respectively
  compute to the numerals \(\numeral n\) and \(\numeral {n+1}\).
\end{exercise}

\begin{exercise}\label{exer:pcf-add-mult}
  Construct PCF programs
  \(\pcfadd,\pcffont{mult} : \pcfnat \pcffun \pcfnat \pcffun \pcfnat\)
  implementing addition and multiplication, respectively.
\end{exercise}

Having general recursion also means that we have non-terminating programs, such
as the following one:

\begin{example}\label{exam:non-termination}
  Define
  \(S \coloneqq \pa*{\lambdadot{\var x : \pcfnat}{\pcfsuc \, \var x}} : \pcfnat
  \pcffun \pcfnat\) and consider the PCF program
  \(M \coloneqq \pcffix_{\pcfnat}(S) : \pcfnat\).  Repeatedly applying the
  small-step operational semantics, we get:
  \begin{alignat*}{3}
    M &= \pcffix_{\pcfnat} \, S \\
    &\smallstep
       S(\pcffix_{\pcfnat} \, S)\\
    &\smallstep
      \pcfsuc(\pcffix_{\pcfnat} \, S)
    &&= \pcfsuc \, M \\
    &\smallstep
      \pcfsuc\pa*{S\pa*{\pcffix_{\pcfnat} \, S}} \\
    &\smallstep
      \pcfsuc\pa*{\pcfsuc\pa*{\pcffix_{\pcfnat} \, S}}
    &&= \pcfsuc\pa*{\pcfsuc\, M} \\
    &\smallstep \dots
  \end{alignat*}
  Thus, the term \(M : \pcfnat\) never computes to a numeral.
\end{example}

\section{Big-step operational semantics}

In our investigations into the denotational semantics of PCF, we will typically
not be interested in intermediate reduction steps, e.g.\ we don't really care
about the whole sequence
\({\pcfifz(\pcfpred\,\numeral{3},\numeral{5},\pcfpred\,\numeral{7})} \smallstep
{\pcfifz(\numeral{2},\numeral{5},\pcfpred\,\numeral{7})} \smallstep
{\pcfpred\,\numeral{7}} \smallstep {\numeral{6}}\); we only care that
\(\pcfifz(\pcfpred\,\numeral{3},\numeral{5},\pcfpred\,\numeral{7})\) computes to
the numeral \(\numeral{6}\).
%
The big-step operational semantics of PCF is a way of formalising this
desideratum.

\begin{definition}[Big-step operational semantics of PCF, \(M \bigstep V\)]
  We inductively define when a term \(M\) \emph{(big-step) reduces} to another
  term \(V\) (of the same type, in the same context), written
  \(M \bigstep V\), by the following inductive clauses:
  \begin{longtable}{cc}
  \AxiomC{\phantom{${\bigstep}$}}
  \UnaryInfC{\(\var x \bigstep \var x\)}
  \DisplayProof
  &
  \AxiomC{\phantom{${\bigstep}$}}
  \UnaryInfC{\(\lambdadot{\var x : \sigma}{M} \bigstep \lambdadot{\var x : \sigma}{M}\)}
  \DisplayProof\vspace{1cm}\\
  \AxiomC{\(M \bigstep \lambdadot{\var x : \sigma}{E}\)}
  \AxiomC{\(E[N/x] \bigstep V\)}
  \BinaryInfC{\(M \, N \bigstep V\)}
  \DisplayProof
  &
  \AxiomC{\(M(\pcffix_{\sigma} \, M) \bigstep V\)}
  \UnaryInfC{\({\pcffix_{\sigma} \, M} \bigstep V\)}
  \DisplayProof\vspace{1cm}\\
  \AxiomC{\phantom{${\bigstep}$}}
  \UnaryInfC{\(\numeral 0 \bigstep \numeral 0\)}
  \DisplayProof
  &
  \AxiomC{\(M \bigstep \numeral{n}\)}
  \UnaryInfC{\(\pcfsuc \, M \bigstep \numeral{n+1}\)}
  \DisplayProof\vspace{1cm}\\
  \AxiomC{\(M \bigstep \numeral 0\)}
  \UnaryInfC{\(\pcfpred \, M \bigstep \numeral 0\)}
  \DisplayProof
  &
  \AxiomC{\(M \bigstep \numeral {n+1}\)}
  \UnaryInfC{\(\pcfpred \, M \bigstep \numeral n\)}
  \DisplayProof\vspace{1cm}\\
  \AxiomC{\(M \bigstep \numeral 0\)}
  \AxiomC{\(N_1 \bigstep V\)}
  \BinaryInfC{\(\pcfifz(M,N_1,N_2) \bigstep V\)}
  \DisplayProof
  &
  \AxiomC{\(M \bigstep \numeral {n+1}\)}
  \AxiomC{\(N_2 \bigstep V\)}
  \BinaryInfC{\(\pcfifz(M,N_1,N_2) \bigstep V\)}
  \DisplayProof\qedhere
  \end{longtable}
\end{definition}

\begin{definition}[Value]
  A term is a \emph{value} if it is either a variable, a numeral or a
  \(\lambda\)-abstraction, i.e.\ it is of the form
  \(\lambdadot{\var x : \sigma}{N}\) for some term \(N\).

  Note that the values of type \(\pcfnat\) in the empty context are precisely
  the numerals.
\end{definition}

The reason that we use \(V\) in the big-step semantics is the following:

\begin{lemma}
  If \(M \bigstep V\), then \(V\) is a value.
\end{lemma}

Moreover, values do not reduce any further:

\begin{lemma}
  If \(V\) is a value, then
  \begin{enumerate}[(i)]
  \item there is no term \(N\) such that \(V \smallstep N\), and
  \item whenever \(V \bigstep N\), we have \(V = N\).
  \end{enumerate}
\end{lemma}

Thus, the big-step operational semantics reduces a term all the way to a value,
forgetting about the intermediary reductions.

Furthermore, the values computed by the big-step operational semantics are
unique:
\begin{lemma}[Big-step is deterministic]
  If \(M \bigstep V\) and \(M \bigstep W\), then \(V = W\).
\end{lemma}

%(Actually, the small-step operational semantics is also deterministic.)

The lemmas are proved by induction on the structure of derivations of
\(M \bigstep V\) and inspection of the small-step operational semantics.

Finally, we relate the small-step and big-step operational semantics:
%
We write \(\smallstep^\ast\) for the reflexive transitive closure of
\(\smallstep\). That is, we have \(M \smallstep^\ast N\) exactly if there is a
sequence \(M = M_0 \smallstep M_1 \smallstep \dots \smallstep M_{n-1} = N\). For
\(n = 0\), this reads \(M = N\), and for \(n = 1\), it reads \(M \smallstep N\).

\begin{exercise}\label{exer:small-and-big-step}
  Show that:
  \begin{enumerate}[(i)]
  \item\label{bigstep-implies-smallstep} if \(M \bigstep V\), then \(M \smallstep^\ast V\);
  \item\label{smallstep-gives-bigstep} if \(M \smallstep N\), then
    \(N \bigstep V\) implies \(M \bigstep V\) for all values \(V\);
  \item\label{smallstep-gives-bigstep'} if \(M \smallstep^\ast N\), then
    \(N \bigstep V\) implies \(M \bigstep V\) for all values \(V\).
  \end{enumerate}
  For~\ref{bigstep-implies-smallstep} and \ref{smallstep-gives-bigstep}, use
  induction on the structure of the derivations; for
  \ref{smallstep-gives-bigstep'} you can repeatedly
  apply~\ref{smallstep-gives-bigstep}.

  Conclude that \(M \bigstep V\) if and only if \(M \smallstep^\ast V\) for all
  terms \(M\) and values \(V\).
\end{exercise}

\section{List of exercises}
\begin{enumerate}
\item \cref{exer:small-step-addition}: On computing additions using the
  small-step operational semantics.
\item \cref{exer:pcf-add-mult}: On defining addition and multiplication in PCF.
\item \cref{exer:small-and-big-step}: On relating the small-step and big-step
  operational semantics.
\end{enumerate}




%%% Local Variables:
%%% mode: latexmk
%%% TeX-master: "../main"
%%% End:

\chapter[{Denotational semantics and domain theory}]{Denotational semantics and \\ domain theory}\label{chap:domains}

Previously we introduced the operational semantics of PCF. Next, we wish to
introduce a \emph{denotational semantics} to PCF. The basic idea is to assign to
each type \(\sigma\) of PCF some kind of \emph{mathematical} object
\(\densem{\sigma}\).
%
A program \(M\) (i.e.\ closed terms) of type \(\sigma\) is then interpreted as
an element \(\densem{M}\) of \(\densem{\sigma}\).
%
More generally, we extend the interpretation of types to contexts and
\(\Gamma \vdash M : \sigma\) will then be interpreted as a certain
(\(\omega\)-continuous) function from \(\densem{\Gamma}\) to
\(\densem{\sigma}\).

For the denotational semantics we have three desiderata:

\begin{itemize}[itemsep=2mm,topsep=3mm]
\item\emph{Compositionality}: This can be summarised as follows:
  \begin{displayquote}
    \emph{The interpretation of a composite is the composite of the interpretations}.
  \end{displayquote}

  %
  For example, the interpretation of a function type \(\sigma \pcffun \tau\)
  will be a certain set of functions from \(\densem{\sigma}\) to
  \(\densem{\tau}\), and if we have programs \(M : \sigma \pcffun \tau\) and
  \(N : \sigma\), then \(\densem{M \, N} = \densem{M}(\densem{N})\).
\item\emph{Soundness}: We want our interpretation to respect the operational
  semantics, i.e.\ if we have terms \(M\) and \(N\) with
  \(M \bigstep N\), then their interpretations should be equal.
\item\emph{Computational adequacy}: We should be able to use our
  interpretation to compute, i.e.\ if \(M\) is a program of type \(\pcfnat\)
  and \(\densem{M} = n\) for some natural number \(n\), then
  \(M \bigstep \numeral{n}\).
\end{itemize}

Soundness and computational adequacy will be proved
in~\cref{sec:soundness,chap:comp-adequacy}, while compositionality will be a
direct consequence of our definitions.

\section{Towards domain theory}

All this leaves the question what kind of mathematical objects we have in mind
for interpreting the types of PCF. A first, naive attempt may be to interpret
each type \(\sigma\) as a set \(\densem{\sigma}\) with:
\begin{align*}
  \densem{\pcfnat} &\coloneqq \text{the set \(\Nat\) of natural numbers}, \text{ and} \\
  \densem{\sigma \pcffun \tau} &\coloneqq \text{the set \(\set{f \colon \densem{\sigma} \to
    \densem{\tau}}\) of all functions from \(\densem{\sigma}\) to \(\densem{\tau}\)}.
\end{align*}

There are two problems with this naive approach.
%
The first problem is that \(\densem{\pcfnat}\) should not only offer natural
numbers, but it should also offer an interpretation of programs of type
\(\pcfnat\) that do not terminate such as the one
in~\cref{exam:non-termination}.
%
Right now, this program \(M\) would have to be interpreted as some natural
number \(n\), forcing us to make some arbitrary choice. Moreover, it is not true
that \(M \bigstep \numeral n\), whatever choice we make, so we would violate
computational adequacy.

This problem is easily solved by introducing a new, special element
\(\bot_\sigma \in \densem{\sigma}\) that interprets non-terminating programs
of type \(\sigma\). And indeed, this will be a part of our eventual interpretation.

However, it does not solve the second, more serious problem, namely that, in
order to interpret PCF's \(\pcffix_\sigma\) we would need every function
\(f \colon \densem{\sigma} \to \densem{\sigma}\) to have a fixed point.
%
But this is easily seen to \emph{false}, e.g. consider
\begin{align*}
  f \colon \Nat \cup \{\bot\} &\to \Nat \cup \{\bot\} \\
  \bot &\mapsto 0, \\
  n &\mapsto n+1.
\end{align*}

This is where domain theory comes to the rescue: Instead of using plain sets, we
will interpret types as so-called \emph{\(\omega\)-cppos}: these are
\emph{partially ordered} sets with a \emph{least} element \(\bot\) and
\emph{least upper bounds} of increasing sequences.
%
A function between (the underlying sets of) two \(\omega\)-cppos is
\emph{\(\omega\)-continuous} if it preserves the order and these least upper
bounds.
%
The upshot of restricting to such \(\omega\)-continuous functions is that these
can be shown to have (least) fixed points, thus solving our second problem.

\section{Basic definitions and the least fixed point theorem}

We give the definitions of \(\omega\)-cppos and \(\omega\)-continuous functions
between them.

\begin{definition}[Poset]
  A \emph{partially ordered set}, or \emph{poset}, is a set \(X\) together with
  a binary relation \({\below}\) satisfying:
  \begin{enumerate}[(i)]
  \item \emph{reflexivity}: for every \(x \in X\), we have \(x \below x\);
  \item \emph{transitivity}: for every \(x,y,z \in X\), if \(x \below y\) and
    \(y \below z\), then \(x \below z\);
  \item \emph{antisymmetry}: for every \(x,y \in X\), if \(x \below y\) and
    \(y \below x\), then \(x = y\). \qedhere
  \end{enumerate}
\end{definition}

\begin{example}
  The natural numbers, rational numbers and real numbers with their usual
  orderings are all examples of posets.
\end{example}

\begin{definition}[\(\omega\)-chain, least upper bound, \(\omega\)-cpo]
  Let \((X,{\below})\) be a poset.
  \begin{enumerate}[(i)]
  \item An \(\omega\)-chain in \(X\) is a subset
    \(\set{x_0,x_1,x_2,\dots} \subseteq X\) of increasing elements, i.e.\ we
    have \(x_0 \below x_1\), \(x_1 \below x_2\), \(x_2 \below x_3\), and so
    on.% \below \dots\).
  \item An \emph{upper bound} of a subset \(S \subseteq X\) is an element
    \(x \in X\) such that \(s \below x\) for every \(s \in S\).
  \item A \emph{least upper bound} of a subset \(S \subseteq X\) is an upper
    bound \(x \in X\) of \(S\) such that for every upper bound \(y \in X\) of
    \(S\) we have \(x \below y\).
  \item The poset \(X\) is \emph{\(\omega\)-chain complete} if every
    \(\omega\)-chain in \(X\) has a least upper bound.% in \(X\).
  \end{enumerate}
  An \(\omega\)-chain complete poset will be called an \emph{\(\omega\)-cpo}.
\end{definition}

\begin{exercise}\label{exer:least-upper-bounds-are-unique}
  Show that least upper bounds are unique. That is, if \(x\) and \(y\) are least
  upper bounds of some subset \(S\) of a poset \(X\), then \(x = y\).

  \emph{Hint}: Use antisymmetry.
\end{exercise}

Because least upper bounds are unique, we will speak of \emph{the} least upper
bound (of a given subset) and in the case of \(\omega\)-chain
\(x_0 \below x_1 \dots\), we will denote the least upper bound by
\(\bigsqcup_{n \in \Nat}x_n\).

\begin{definition}[Least element, \(\omega\)-cppo]
  Let \((X,{\below})\) be a poset.
  \begin{enumerate}[(i)]
  \item A \emph{least element} of \(X\) is an element \(x \in X\) such that
    \(x \below y\) for every \(y \in X\).
  \item An \(\omega\)-cpo is called \emph{pointed} if it has a least element.
  \end{enumerate}
  A pointed \(\omega\)-cpo will be called an \emph{\(\omega\)-cppo}.
\end{definition}

\begin{exercise}\label{exer:least-element-is-unique}
  Let \(X\) be a poset with a least element \(x\). Show that \(x\) is the least
  upper bound of the empty subset and hence conclude that \(x\) must be unique.
\end{exercise}

Because least elements are unique, we will speak of \emph{the} least element and
will typically denote it by \(\bot\).

\begin{example}[\(\Nat_\bot\)]\label{exam:N_bot}
  A fundamental example of an \(\omega\)-cppo is the \emph{flat} \(\omega\)-cppo
  \(\Nat_\bot\), which is given by the set \(\Nat \cup \{\bot\}\) ordered as
  depicted in the following diagram:
  \[
    \begin{tikzcd}
      0 & 1 & 2 & 3 & \ldots \\
      & & \bot \ar[ull,no head] \ar[ul,no head] \ar[u,no head] \ar[ur, no head]
      \ar[urr, no head]
    \end{tikzcd}
  \]
  Here we interpret a line going up from an element \(x\) to an element \(y\) as
  saying that \(x \below y\).
  %
  Thus, the element \(\bot\) is the least element and all other elements are
  unrelated.
  %
  In particular, we have do \emph{not} have \(0 \below 1\) as in the usual
  ordering of the natural numbers.

  The intuition here is that the partial order \({\below}\) does not reflect the
  numerical value, but rather how ``defined'' an element is with \(\bot\)
  representing an ``undefined''.
  %
  In our interpretation of PCF, we will interpret \(\pcfnat\) as \(\Nat_\bot\)
  and \(\bot\) will serve as an interpretation of non-terminating programs.
\end{example}

\begin{exercise}\label{exer:N_bot}
  Show that the \(\omega\)-chains in \(\Nat_\bot\) are precisely \(\set{\bot}\),
  \(\set{n}\) and \(\set{\bot,n}\) for a natural number \(n \in \Nat\).
  %
  Use this to check that \(\Nat_\bot\) is indeed an \(\omega\)-cpo.
\end{exercise}

\begin{remark}
  It is perhaps a bit arbitrary that to exhibit
  \(\set{\bot,n} \subseteq \Nat_\bot\) as an \(\omega\)-chain we have to repeat
  elements and order them linearly.
  %
  One way to address this is to replace the notion of \(\omega\)-chain by that
  of a \emph{directed} (see~\cref{exer:directedness}) subset, and to consider
  \emph{dcpos}: posets with least upper bounds for all directed subsets.

  For general domain theory the notion of dcpo is arguably the better notion
  (see also~\cite[Section~2.2.4]{AbramskyJung1994}, but for the denotational
  semantics of PCF and in particular, the existence of least fixed
  points~(\cref{least-fixed-point}), the notion of \(\omega\)-cpo suffices.
\end{remark}

\begin{exercise}\label{exer:directedness}
  Let \((X,{\below})\) be a poset. A subset \(S \subseteq X\) is \emph{directed}
  if it is non-empty and for every two elements \(x,y \in S\), there exists a
  element \(z \in S\) with \(x \below z\) and \(y \below z\).
  \begin{enumerate}[(i)]
  \item Show that every \(\omega\)-chain is a directed subset.
  \item Conversely, use the axiom of choice to show that every countable,
    directed subset is an \(\omega\)-chain.\qedhere
  \end{enumerate}
\end{exercise}

We proceed by defining the \(\omega\)-continuous maps between (pointed)
\(\omega\)-cpos.
\begin{definition}[\(\omega\)-continuity]\label{def:continuity}
  A function \(f \colon A \to B\) between the underlying sets of two
  \(\omega\)-cpos \(A\) and \(B\) is \emph{\(\omega\)-continuous} if
  \begin{enumerate}[(i)]
  \item\label{monotone} it is \emph{monotone} (or \emph{order preserving}),
    i.e.\ if \(x \below y\) in \(A\), then \(f(x) \below f(y)\) in \(B\);
  \item\label{preserve-lub-of-omega-chains} it \emph{preserves least upper
      bounds of \(\omega\)-chains}, i.e.\ if \(\set{x_0,x_1,\dots}\) is an
    \(\omega\)-chain in \(A\) with least upper bound \(a\), then \(f(a)\) is the
    least upper bound in \(B\) of the subset \(\set{f(x_0),f(x_1),\dots}\). \qedhere
  \end{enumerate}
\end{definition}

\begin{remark}
  It follows from monotonicity that \(\set{f(x_0),f(x_1),\dots}\) is an
  \(\omega\)-chain in \(B\) and hence we could rewrite the second condition as:
  \[
    f\pa*{\textstyle\bigsqcup_{n \in \Nat}x_n} = \textstyle\bigsqcup_{n \in
      \Nat}f(x_n). \qedhere
  \]
\end{remark}

It is worthwhile to reflect on the computational intuitions underlying
monotonicity and preservation of least upper bounds of \(\omega\)-chains.
%
If we think of \(f \colon A \to B\) as some computational procedure, then
monotonicity says:
\begin{displayquote}
  \emph{more (or better) input leads to more (or better) output},
\end{displayquote}
while the second condition of \(\omega\)-continuity says:
\begin{displayquote}
  \emph{every output can be patched together from the outputs of approximations}.
\end{displayquote}
The idea here is that a computational procedure can only inspect a finite amount
of input before returning an output, so that the approximations suffice to
determine the output.
%
These intuitions and ideas are nicely illustrated in the following example:
\begin{exercise}\label{exer:cantor-domain}
  Let \(C\) be the poset of finite and infinite binary sequences ordered by
  prefix. For an infinite sequence \(\alpha\) we write
  \(\alpha \upharpoonright n\) for its finite prefix of length \(n\).

  \begin{enumerate}[(i)]
  \item Show that \(C\) is an \(\omega\)-cppo.
  \item Show that a function \(f \colon C \to C\) is \(\omega\)-continuous if
    and only if for every infinite sequence \(\alpha\) we have
    \(f(\alpha) = \textstyle\bigsqcup_{n \in \Nat}f(\alpha \upharpoonright n)\). \qedhere
  \end{enumerate}
\end{exercise}

% or alternatively, if we think of \(x \below y\) as expressing that \(y\) is at
% least as informative as \(x\), then \(f(y)\) is at least as informative as
% \(f(x)\).

Actually, monotonicity, item~\ref{monotone} in~\cref{def:continuity}, is
redundant as the following exercise ask you to check:

\begin{exercise}\label{exer:monotonicity-follows}
  Show that if a function \(f\) between \(\omega\)-cpos satisfies
  item~\ref{preserve-lub-of-omega-chains} of~\cref{def:continuity}, then \(f\)
  must be monotone.

  \emph{Hint}: If \(x \below y\), then \(\set{x,y}\) is an \(\omega\)-chain with
  least upper bound \(y\).
\end{exercise}

\begin{example}
  The function \(f \colon \Nat_\bot \to \Nat_\bot\) given by \(\bot \mapsto 0\)
  and \(n \mapsto n + 1\) is \emph{not} \(\omega\)-continuous, because it is not
  monotone: \(\bot \below 1\), but \(f(\bot) = 0\) is not below \(f(1) = 2\) in
  the ordering of \(\Nat_\bot\).

  On the other hand, the function \(g \colon \Nat_\bot \to \Nat_\bot\) given by
  \(\bot \mapsto \bot\) and \(n \mapsto n+1\) is \(\omega\)-continuous.
  %
  In fact, it will serve as the interpretation of the PCF program
  \(\lambdadot{\var x : \pcfnat}{\pcfsuc\,\var x}\) of type
  \(\pcfnat \pcffun \pcfnat\).
\end{example}

The point of introducing \(\omega\)-continuity is the following fundamental
result:

\begin{theorem}[Knaster--Tarski fixed point theorem]\label{least-fixed-point}
  Every \(\omega\)-continuous function \(f \colon X \to X\) on an \(\omega\)-cpo
  \(X\) has a least fixed point given by the least upper bound of the
  \(\omega\)-chain
  \[
    \bot \below f(\bot) \below f(f(\bot)) \dots.
  \]
\end{theorem}
\begin{proof}[Proof sketch]
  We write \(f^n(\bot)\) for the \(n\)-fold application of \(f\) to the least
  element \(\bot\).
  %
  We first check that \(\set{f^n(\bot) \mid n \in \Nat}\) is indeed an
  \(\omega\)-chain.
  %
  Since \(\bot\) is the least element, we certainly have
  \(\bot \below f(\bot)\). But now we also have \(f(\bot) \below f^2(\bot)\) by
  monotonicity of \(f\). It follows by induction on \(n\) that
  \(f^n(\bot) \below f^{n+1}(\bot)\), as desired.
  %
  Now we calculate:
  \begin{align*}
    f\pa*{\textstyle\bigsqcup_{n \in \Nat}f^n(\bot)}
    &= \bigsqcup_{n \in \Nat}f\pa*{f^n(\bot)}
    &\text{(since \(f\) preserves least upper bounds of \(\omega\)-chains)} \\
    &= \bigsqcup_{n \in \Nat}f^{n+1}(\bot)
    &\text{(by definition)} \\
    &= \bigsqcup_{n \in \Nat}f^{n}(\bot)
    &\text{(by antisymmetry and calculation)}
  \end{align*}
  so we have a fixed point, as claimed. That it is the least follows
  from~\cref{exer:least-fixed-point}.
\end{proof}

\begin{exercise}\label{exer:least-fixed-point}
  Write \(x_0\) for the fixed point constructed in~\cref{least-fixed-point}.
  %
  Show that for every \(y \in X\) with \(f(y) \below y\), we have \(x_0 \below y\).
  %
  Conclude that \(x_0\) is indeed the \emph{least} fixed point.
\end{exercise}

\begin{exercise}\label{exer:category-of-cpos}
  Show that \(\omega\)-cpos and \(\omega\)-continuous functions form a
  \emph{category}, i.e.\
  \begin{enumerate}
  \item if \(X\) is an \(\omega\)-cpo, then the identity function
    \(x \mapsto x\) on \(X\) is \(\omega\)-continuous, and
  \item if we have \(\omega\)-continuous functions \(f \colon A \to B\) and
    \(g \colon B \to C\), then so is their composition (as functions)
    \(g \circ f \colon A \to C\). \qedhere
  \end{enumerate}
\end{exercise}

\begin{exercise}[For those familiar with category theory]\label{exer:adjunctions}

  Write \(\omegaCPO\) for the category constructed above and \(\omegaCPPObot\)
  for the category of \emph{pointed} \(\omega\)-cpos and those morphisms of
  \(\omega\)-cpos that preserve the least element.

  Construct adjunctions
  \[
  \begin{tikzcd}
    \Set
    \arrow[r, ""{name=F}, bend left=35] &
    \omegaCPO
    \arrow[l, ""{name=G}, bend left=25]
    \arrow[phantom, from=F, to=G, "\dashv" rotate=-90]
    \arrow[r, ""{name=F}, bend left=25] &
    \omegaCPPObot
    \arrow[l, ""{name=G}, bend left=20]
    \arrow[phantom, from=F, to=G, "\dashv" rotate=-90]
  \end{tikzcd}
\]
\emph{Hint}: The composite functor \(\Set \to \omegaCPPObot\) takes the set \(\Nat\) to
the \(\omega\)-cppo \(\Nat_\bot\) from~\cref{exam:N_bot}.
\end{exercise}

\section{Products: interpreting contexts}

As explained in the introduction to this chapter our goal is to interpret the
types of PCF as \(\omega\)-cppos. Thus, we will have an \(\omega\)-cppo
\(\densem{\sigma}\) for each PCF type \(\sigma\).
%
Moreover, we wish to extend this interpretation to contexts, which are lists of
typed variables. In a context
\(\Gamma = [\var x_0 : \sigma_0, \var x_1 : \sigma_1 , \dots , \var x_n :
\sigma_{n-1}]\) the variables themselves are not really important; it's the
types that matter. Accordingly, we will interpret such a context as the
\emph{product}
\(\densem{\sigma_0} \times \densem{\sigma_1} \times \cdots \times
\densem{\sigma_{n-1}}\) of the interpretations of the types.

The empty context will be interpreted as follows:

\begin{example}[The one point \(\omega\)-cppo \(\One\)]\label{exam:one-point}
  The one point \(\omega\)-cppo \(\One\) is the unique \(\omega\)-cppo with the
  singleton \(\set{\star}\) as its underlying set.
\end{example}

\begin{definition}[Binary product of \(\omega\)-cpos, \(A \times B\)]
  Given two \(\omega\)-cpos \(A\) and \(B\), their \emph{binary product}
  \(A \times B\) is defined by taking the cartesian product of their underlying
  sets with pairwise partial order:
  \[
    (x_1 , y_1) \below_{A \times B} (x_2 , y_2) %
    \coloneqq (x_1 \below_A x_2) \text{ and } (y_1 \below_B y_2).
  \]
  Given an \(\omega\)-chain \(\set{(x_0,y_0),(x_1,y_1),\dots}\) in
  \(A \times B\), one verifies that \(\set{x_0,x_1,\dots}\) and
  \(\set{y_0,y_1,\dots}\) are \(\omega\)-chains in \(A\) and \(B\),
  respectively.
  %
  Hence, we can consider their least upper bounds \(\bigsqcup_{n \in \Nat}x_n\)
  and \(\bigsqcup_{n \in \Nat}y_n\) in \(A\) and \(B\).
  %
  One then checks that the least upper bound of the original \(\omega\)-chain in
  \(A \times B\) is given by the pair of least upper bounds
  \(\pa*{\bigsqcup_{n \in \Nat}x_n , \bigsqcup_{n \in \Nat}y_n}\).

  Finally, if \(A\) and \(B\) are pointed, then so is \(A \times B\) with least
  element \(\pa*{\bot_A , \bot_B}\).
\end{definition}

\begin{example}
  The product of the \(\omega\)-cppos %
  \(\begin{tikzcd}[row sep=8pt]
    x \\
    \bot \ar[u, no head]
  \end{tikzcd}\)
  and \( \begin{tikzcd}[row sep=8pt]
    y \\
    \bot \ar[u, no head]
  \end{tikzcd}
  \) looks like this:

  \[
    \begin{tikzcd}[row sep=15pt,column sep=1pt]
      & (x,y) \\
      (x,\bot) \ar[ur, no head]& & (\bot,y) \ar[ul, no head] \\
      & (\bot,\bot) \ar[ur,no head] \ar[ul, no head]
    \end{tikzcd}\qedhere
  \]
\end{example}

\begin{example}[An illustration of the \(\omega\)-cppo \(\Nat_\bot \times \Nat_\bot\)]
  \[
    \begin{tikzcd}[column sep=1pt, row sep=50pt]
      & (0,0) & (0,1) & (1,0) & (1,1) & \dots \\
      (\bot,0) \ar[ur, no head, shift right=5pt]
      \ar[urrr, no head, shift right=5pt]
      %\ar[urrrrr, no head, shift right=5pt]
      & (\bot,1)
      \ar[ur, no head, shift right=5pt]
      \ar[urrr, no head, shift right=5pt]
      %\ar[urrrr, no head, shift right=5pt]
      & \dots
      &
      & (0,\bot)
      \ar[ulll, no head, shift left=5pt]
      \ar[ull, no head, shift left=5pt]
      & (1,\bot)
      \ar[ull, no head, shift left=5pt]
      \ar[ul, no head, shift left=5pt]
      & \dots \\
      & & & (\bot,\bot)
      %
      \ar[ulll,no head, shift left=5pt] \ar[ull, no head, shift left=5pt]
      \ar[ul, no head, shift left=5pt] \ar[ur, no head, shift right=5pt]
      \ar[urr, no head, shift right=5pt] \ar[urrr, no head, shift right=5pt]
    \end{tikzcd}\qedhere
  \]
\end{example}

The construction of least upper bounds of \(\omega\)-chains in the product
ensures that we have the following result:
\begin{lemma}\label{projections}
  Given two \(\omega\)-cpos \(A\) and \(B\), the \emph{projections}
  \begin{align*}
    \pi_1 \colon A \times B &\to A &&& \pi_2 \colon A \times B &\to B \\
    (x,y) &\mapsto x &&& (x,y) &\mapsto y
  \end{align*}
  are \(\omega\)-continuous.
\end{lemma}

\begin{exercise}\label{exer:continuity-in-each-argument}
  Show that a function \(f \colon A \times B \to C\) between \(\omega\)-cpos is
  \(\omega\)-continuous if and only if the functions \(f(x,-) \colon B \to C\) and
  \(f(-,y) \colon A \to C\) are \(\omega\)-continuous for every \(x \in A\) and
  \(y \in B\).

  Thus, \(\omega\)-continuity of \(f\) can be checked in each argument
  separately.
\end{exercise}

\begin{exercise}\label{exer:product-induced}
  Prove that if \(f \colon C \to A\) and \(g \colon C \to B\) are
  \(\omega\)-continuous functions between \(\omega\)-cpos, then so is the
  induced map
  \begin{align*}
    \langle f , g \rangle  \colon C &\to A \times B \\
    x &\mapsto (f(x),g(x)).
  \end{align*}

  For those familiar with category theory: conclude that the category
  \(\omegaCPO\) has finite products.
\end{exercise}

\section{Exponentials: interpreting function types}

To interpret the function type \(\sigma \Rightarrow \tau\) of PCF using
\(\omega\)-cpos, we are going to construct an \(\omega\)-cpo of
\(\omega\)-continuous functions: the \emph{exponential}.

\begin{definition}[Exponential of \(\omega\)-cpos, \(B^A\)]\label{def:exponential}
  Given two \(\omega\)-cpos \(A\) and \(B\), their \emph{exponential} \(B^A\) is
  defined by equipping the set of \(\omega\)-continuous functions from
  \(A\)~to~\(B\) with the \emph{pointwise} order:
  \[
    f \below g
    \coloneqq \forall_{x \in A} \, \pa{f(x) \below_B g(x)}.
  \]
  Given an \(\omega\)-chain \(\set{f_0,f_1,\dots}\) in \(B^A\), one verifies
  that \(\set{f_0(x),f_1(x),\dots}\) is an \(\omega\)-chain in \(B\) for every
  \(x \in A\).
  %
  Hence, for every \(x \in A\), we can consider the least upper bound
  \(\bigsqcup_{n \in \Nat}f_n(x)\) in \(B\).
  %
  One then checks that the function \(x \mapsto \bigsqcup_{n \in \Nat}f_n(x)\)
  is \(\omega\)-continuous and moreover, that it is the least upper bound of the
  original \(\omega\)-chain in \(B^A\).

  Finally, if \(B\) is pointed, then so is \(B^A\) with least
  element \(x \mapsto \bot\).
\end{definition}

We illustrate the exponential with the following example and exercise before
proceeding to develop the required machinery for interpreting PCF using
\(\omega\)-cppos and \(\omega\)-continuous maps.

\begin{example}
  Consider the \(\omega\)-cpo \(\Two \coloneqq
  \begin{tikzcd}[row sep=5pt]
    \top \ar[d,no head] \\ \bot
  \end{tikzcd}\). Writing \([x,y]\) for the function
  \({f \colon \Two \to \Two}\) given by \(f(\bot) \coloneqq x\) and
  \(f(\top) \coloneqq y\), we picture the exponential \(\Two^\Two\) as:
  \[
  \begin{tikzcd}[row sep=10pt]
    {[\top,\top]} \ar[d, no head] \\
    {[\bot,\top]} \ar[d, no head] \\
    {[\bot,\bot]}
  \end{tikzcd}
  \]
  Note that \([\top,\bot]\) is \emph{not} an element of \(\Two^\Two\), because
  it isn't monotone.
\end{example}

\begin{exercise}\label{exer:approximating-successor}
  For each natural number \(k \in \Nat\), define the function
  \(s_k \colon \Nat_\bot \to \Nat_\bot\) by
  \[
    s_k(\bot) \coloneqq \bot
    \quad\text{and}\quad
    s_k(n) \coloneqq
    \begin{cases}
      n+1 &\text{if \(n \leq k\)},
      \\ \bot &\text{else}.
    \end{cases}
  \]
  \begin{enumerate}[(i)]
  \item Show that each \(s_k\) is \(\omega\)-continuous.
  \item Show that \(\set{s_k \mid k \in \Nat}\) is an \(\omega\)-chain in
    \(\Nat_\bot^{\Nat_\bot}\) and its least upper bound is
    \(s \colon \Nat_\bot \to \Nat_\bot\) with \(s(\bot) \coloneqq \bot\) and
    \(s(n) \coloneqq n+1\).
    % \begin{align*}
    %   s \colon \Nat_\bot &\to \Nat_\bot \\
    %   \bot &\mapsto \bot \\
    %   n &\mapsto n+1.
    % \end{align*}
  \end{enumerate}
  Thus, the functions \((s_k)_{k \in \Nat}\) are increasingly better approximations of the
  successor map on \(\Nat_\bot\).
\end{exercise}

\begin{lemma}\label{evaluation-is-continuous}
  Given two \(\omega\)-cpos \(A\) and \(B\), the \emph{evaluation} function
  \begin{align*}
    B^A \times A &\to B \\
    (f,x) &\mapsto f(x)
  \end{align*}
  is \(\omega\)-continuous.
\end{lemma}
\begin{exercise}\label{exer:evaluation-is-continuous}
  Prove~\cref{evaluation-is-continuous}.

  \emph{Hint}: Use~\cref{exer:continuity-in-each-argument}.
\end{exercise}

\begin{exercise}\label{exer:curry-is-continuous}
  Show that if \(f \colon C \times A \to B\) is an \(\omega\)-continuous
  function between \(\omega\)-cpos, then so is the \emph{curried} function
  \begin{align*}
    \hat{f} \colon C &\to B^A \\
    \hat{f}(c) &\coloneqq a \mapsto f(c,a).
  \end{align*}

  For those familiar with category theory: conclude that the category
  \(\omegaCPO\) is cartesian closed.
\end{exercise}

The following theorem says that the construction of the Knaster--Tarski least
fixed point (\cref{least-fixed-point}) is continuous and will be used to
interpret PCF's fixed point operator
\(\pcffix_\sigma : \pa*{\sigma \pcffun \sigma} \pcffun \sigma\).
\begin{theorem}\label{least-fixed-point-is-continuous}
  For every \(\omega\)-cpo \(A\), the assignment
  \begin{align*}
    \mu \colon A^A &\to A \\
    f &\mapsto \textstyle\bigsqcup_{n \in \Nat}f^n(\bot)
  \end{align*}
  of an \(\omega\)-continuous endofunction on \(A\) to its least fixed point is
  \(\omega\)-continuous.
\end{theorem}
\begin{proof}[Proof sketch]
  For each natural number \(n \in \Nat\), define
  \begin{align*}
    \iter_n \colon A^A &\to A \\
    f &\mapsto f^n(\bot).
  \end{align*}
  By induction on \(n\), we see that each \(\iter_n\) is \(\omega\)-continuous.
  %
  Thus, they are elements of the exponential \(A^{\pa{A^A}}\).
  %
  Moreover, \(\set{\iter_n \mid n \in \Nat}\) is an \(\omega\)-chain in this
  \(\omega\)-cpo.
  %
  Hence, we can take its least fixed point \(F \in A^{\pa{A^A}}\) which is an
  \(\omega\)-continuous function by construction.
  %
  Finally, from the construction of least upper bounds in the exponential, we
  calculate that \(F(f) = \mu(f)\) for every \(f \in A^A\), so that \(\mu\)
  must be \(\omega\)-continuous.
\end{proof}

\section{List of exercises}
\begin{enumerate}
\item \cref{exer:least-upper-bounds-are-unique}: On the uniqueness of least upper bounds.
\item \cref{exer:least-element-is-unique}: On the uniqueness of least elements.
\item \cref{exer:N_bot}: On \(\omega\)-chains in \(\Nat_\bot\).
\item \cref{exer:directedness}: On directed subsets and \(\omega\)-chains.
\item \cref{exer:monotonicity-follows}: On deriving monotonicity from preservation of
  least upper bounds of \(\omega\)-chains.
\item \cref{exer:least-fixed-point}: On showing that the Knaster--Tarski fixed point
  is the least.
\item \cref{exer:category-of-cpos}: On the category of \(\omega\)-cpos.
\item \cref{exer:adjunctions}: On adjunctions between the categories of sets and
  (pointed) \(\omega\)-cpos.
\item \cref{exer:continuity-in-each-argument}: On checking \(\omega\)-continuity
  in each argument of the product.
\item \cref{exer:product-induced}: On \(\omega\)-continuity of the induced map
  to the product.
\item \cref{exer:approximating-successor}: On approximating the successor map on
  \(\Nat_\bot\).
\item \cref{exer:evaluation-is-continuous}: On \(\omega\)-continuity of the evaluation map.
\item \cref{exer:curry-is-continuous}: On \(\omega\)-continuity of the curried map.
\end{enumerate}

%%% Local Variables:
%%% mode: latexmk
%%% TeX-master: "../main"
%%% End:

\chapter{The Scott model of PCF}

We started these notes by introducing PCF and its operational semantics
in~\cref{chap:PCF}. We then developed sufficient domain theory
in~\cref{chap:domains} to give a denotational semantics of PCF, where we
interpret the types of PCF as \(\omega\)-cppos and terms as
\(\omega\)-continuous maps. This chapter is devoted to precisely defining that
interpretation, known as the Scott model of PCF.

\begin{definition}[Interpretation of PCF types]
  We inductively assign an \(\omega\)-cppo \(\densem{\sigma}\) to each type
  \(\sigma\) of PCF:
  \begin{alignat*}{3}
    &\densem{\pcfnat} &&\coloneqq \Nat_\bot, \text{and} \\
    &\densem{\sigma \pcffun \tau} &&\coloneqq \densem{\tau}^{\densem{\sigma}},
  \end{alignat*}
  where we recall \(\Nat_\bot\) from~\cref{exam:N_bot} and the exponential
  from~\cref{def:exponential}.
\end{definition}

This interpretation extends to contexts by using iterated binary products.
\begin{definition}[Interpretation of contexts]
  A context
  \(\Gamma = [\var x_0 : \sigma_0, \dots , \var x_{n-1} :
  \sigma_{n-1}]\)
  is interpreted as
  \[
    \densem{\Gamma} \coloneqq \densem{\sigma_0} \times \cdots \times
    \densem{\sigma_{n-1}}
  \]
  with the convention that the empty context is interpreted as the empty
  product: the one point \(\omega\)-cppo \(\One\) from~\cref{exam:one-point}.
\end{definition}

The interpretation of the terms of PCF proceeds by yet another inductive
definition:

\begin{definition}[Interpretation of PCF terms]\label{def:interpretation}
  A term \(\Gamma \vdash M : \sigma\) of type \(\sigma\) in context~\(\Gamma\)
  will be interpreted as an \emph{\(\omega\)-continuous} function
  \[
    \densem{\Gamma} \xrightarrow{\densem{M}} \densem{\sigma}
  \]
  according to the following clauses which mirror~\cref{def:PCF-terms}:
  \begin{enumerate}[(i)]
  \item
    The interpretation of
    \[
      \def\fCenter{\ \vdash\ }
      \AxiomC{\phantom{$\fCenter$}}
      \UnaryInf$\Gamma,\var{x}:\sigma,\Delta \fCenter \var{x} : \sigma$
      \DisplayProof
    \]
    is
    \[
      \densem{\Gamma} \times \densem{\sigma} \times {\densem{\Delta}}
      \xrightarrow{\pi} \densem{\sigma},
    \]
    where \(\pi\) is a suitable composition of projections
    (recall~\cref{projections}), which is \(\omega\)-continuous, because
    \(\omega\)-continuous functions are closed under composition
    (recall~\cref{exer:category-of-cpos}).
  \item
    To interpret
    \[
      \def\fCenter{\ \vdash\ }
      \Axiom$\Gamma , \var{x} : \sigma \fCenter M : \tau$
      \UnaryInf$\Gamma \fCenter (\lambdadot{\var{x} : \sigma}{M}) : \sigma
      \pcffun \tau$ \DisplayProof
    \]
    we first remark that we have
    \[
      \densem{\Gamma} \times \densem{\sigma} \xrightarrow{\densem{M}} \densem{\tau}
    \]
    by induction hypothesis, so that we can define
    \[
      \densem{\Gamma} \xrightarrow{\densem{\lambdadot{\var x : \sigma}{M}}}
      \densem{\tau}^{\densem{\sigma}}
    \]
    as the curried (recall~\cref{exer:curry-is-continuous}) version of
    \(\densem{M}\), i.e.\ for \(\gamma \in \densem{\Gamma}\), we have
    \(\densem{\lambdadot{\var x : \sigma}{M}}(\gamma) \coloneqq x \mapsto
    \densem{M}(\gamma,x)\).
  \item
    To interpret
    \[
      \def\fCenter{\ \vdash\ }
      \Axiom$\Gamma \fCenter M : \sigma \pcffun \tau$
      \Axiom$\Gamma \fCenter N : \sigma$
      \BinaryInf$\Gamma \fCenter M \, N : \tau$
      \DisplayProof
    \]
    we first remark that we have
    \[
      \densem{\Gamma} \xrightarrow{\densem{M}} \densem{\tau}^{\densem{\sigma}}
      \quad\text{and}\quad
      \densem{\Gamma} \xrightarrow{\densem{N}} \densem{\sigma}
    \]
    by induction hypothesis, so that we can define
    \[
      \densem{\Gamma} \xrightarrow{\densem{M \, N}} \densem{\tau}
    \]
    by application: for \(\gamma \in \densem{\Gamma}\), we have
    \(\densem{M \, N}(\gamma) \coloneqq
    \densem{M}(\gamma)\pa*{\densem{N}(\gamma)}\).

    More abstractly, it is the composition of
    \[
      \densem{\Gamma} \xrightarrow{\langle \densem{M} , \densem{N} \rangle}
      \densem{\tau}^{\densem{\sigma}} \times \densem{\sigma}
      \xrightarrow{\text{evaluation}}
      \densem{\tau},
    \]
    where we recall~\cref{exer:product-induced,evaluation-is-continuous}.
  \item
    To interpret
    \[
      \def\fCenter{\ \vdash\ }
      \Axiom$\Gamma \fCenter M : \sigma \pcffun \sigma$
      \UnaryInf$\Gamma \fCenter \pcffix_\sigma(M) : \sigma$ \DisplayProof
    \]
    we first remark that we have
    \[
      \densem{\Gamma} \xrightarrow{\densem{M}} \densem{\sigma}^{\densem{\sigma}}
    \]
    by induction hypothesis, so that we can define
    \[
      \densem{\Gamma} \xrightarrow{\densem{\pcffix_\sigma\,M}} \densem{\sigma}
    \]
    as the composition
    \(\densem{\Gamma} \xrightarrow{\densem{M}} \densem{\sigma}^{\densem{\sigma}}
    \xrightarrow{\mu} \densem{\sigma}\), where we recall \(\mu\), which assigns
    least fixed points, from~\cref{least-fixed-point-is-continuous}.
  \item
    The interpretation of
    \[
      \def\fCenter{\ \vdash\ }
      \AxiomC{}
      \UnaryInf$\Gamma \fCenter \pcfzero : \pcfnat$
      \DisplayProof
    \]
    is
    \[
      \densem{\Gamma} \xrightarrow{\densem{\pcfzero}} \Nat_\bot
    \]
    which is given by \(\gamma \in \densem{\Gamma} \mapsto 0 \in \Nat_\bot\).
  \item\label{def:interpretation-succ}
    To interpret
    \[
      \def\fCenter{\ \vdash\ }
      \Axiom$\Gamma \fCenter M : \pcfnat$
      \UnaryInf$\Gamma \fCenter \pcfsuc(M) : \pcfnat$
      \DisplayProof
    \]
    we first remark that we have
    \[
      \densem{\Gamma} \xrightarrow{\densem{M}} \Nat_\bot
    \]
    by induction hypothesis, so that we can define
    \[
      \densem{\Gamma} \xrightarrow{\densem{\pcfsuc\,M}} \Nat_\bot
    \]
    as the composition
    \(\densem{\Gamma} \xrightarrow{\densem{M}} \Nat_\bot \xrightarrow{s}
    \Nat_\bot\), where \(s : \Nat_\bot \to \Nat_\bot\) is the successor function
    on \(\Nat_\bot\), i.e.\ \(s(\bot) \coloneqq \bot\) and
    \(s(n) \coloneqq n+1\).
  \item\label{def:interpretation-pred}
    To interpret
    \[
      \def\fCenter{\ \vdash\ }
      \Axiom$\Gamma \fCenter M : \pcfnat$
      \UnaryInf$\Gamma \fCenter \pcfpred(M) : \pcfnat$
      \DisplayProof
    \]
    we first remark that we have
    \[
      \densem{\Gamma} \xrightarrow{\densem{M}} \Nat_\bot
    \]
    by induction hypothesis, so that we can define
    \[
      \densem{\Gamma} \xrightarrow{\densem{\pcfpred\,M}} \Nat_\bot
    \]
    as the composition
    \(\densem{\Gamma} \xrightarrow{\densem{M}} \Nat_\bot \xrightarrow{p}
    \Nat_\bot\), where \(p : \Nat_\bot \to \Nat_\bot\) is the predecessor
    function on \(\Nat_\bot\), i.e.\ \(p(\bot) \coloneqq \bot\),
    \(p(0) \coloneqq 0\) and \(p(n+1) \coloneqq n\).
  \item\label{def:interpretation-ifzero}
    Finally, to interpret
    \[
      \def\fCenter{\ \vdash\ }
      \Axiom$\Gamma \fCenter M : \pcfnat$
      \Axiom$\Gamma \fCenter N_1 : \pcfnat$
      \Axiom$\Gamma \fCenter N_2 : \pcfnat$
      \TrinaryInf$\Gamma \fCenter \pcfifz(M,N_1,N_2) : \pcfnat$
      \DisplayProof
    \]
    we first remark that we have
    \[
      \densem{\Gamma} \xrightarrow{\densem{M}} \Nat_\bot,
      \densem{\Gamma} \xrightarrow{\densem{N_1}} \Nat_\bot \text{ and }
      \densem{\Gamma} \xrightarrow{\densem{N_2}} \Nat_\bot
    \]
    by induction hypothesis, so that we can define
    \[
      \densem{\Gamma} \xrightarrow{\densem{\pcfifz(M,N_1,N_2)}} \Nat_\bot
    \]
    as the composition
    \(\densem{\Gamma}
    \xrightarrow{\langle{\densem{M},\densem{N_1},\densem{N_2}}\rangle}
    {{\Nat_\bot} \times {\Nat_\bot} \times {\Nat_\bot}} \xrightarrow{c}
    \Nat_\bot\), where
    \[
      c : \Nat_\bot \times \Nat_\bot \times \Nat_\bot \to \Nat_\bot
    \] is defined as the \(\omega\)-continuous function
    \(c(\bot,y,z) \coloneqq \bot\), \(c(0,y,z) \coloneqq y\) and
    \(c(n+1,y,z) \coloneqq z\).
  \end{enumerate}
\end{definition}

\begin{remark}[Programs of type \(\sigma\) are interpreted as elements of
  \(\densem{\sigma}\)]
  Note that a program, i.e.\ a closed term, \({} \vdash M : \sigma\) is
  interpreted as a function
  \(\densem{M} \colon \One \to \densem{\sigma}\).
  %
  Since the underlying set of \(\One\) is the singleton \(\set{\star}\), this
  function simply picks out an element of \(\densem{\sigma}\).
  %
  With this in mind, we see that programs of type \(\sigma\) are interpreted
  as elements of \(\densem{\sigma}\).
\end{remark}


\begin{exercise}\label{exer:interpretation-of-numerals}
  Show that the numerals of PCF are interpreted as the natural numbers in
  \(\Nat_\bot\), i.e.\ \(\densem{\numeral{n}} = n \in \Nat_\bot\) for every
  natural number \(n \in \Nat\).
\end{exercise}

\section{Towards soundness}

The next section will be devoted to proving the \emph{soundness} theorem, which
says that if \(M \bigstep N\), then \(\densem{M} = \densem{N}\). In other words,
the interpretation respects the operational semantics.
%
The proof of soundness relies on the following lemma:% , which is actually a
% particular case of the fact that \(M \smallstep N\) implies
% \(\densem{M} = \densem{N}\).

\begin{lemma}[\(\beta\)-equality]\label{beta-equality}
  The Scott model of PCF validates \(\beta\)-equality. That is, if
  \(\Gamma,\var x : \sigma \vdash M : \tau\) and \(\Gamma \vdash N : \sigma\),
  then
  \[
    \densem*{\pa*{\lambdadot{\var x : \sigma}{M}}\,N} = \densem{M[N/\var x]}
  \]
\end{lemma}

\begin{exercise}\label{exer:beta-equality}
  Prove~\cref{beta-equality} from the substitution lemma given below.
\end{exercise}


Suppose that we have a term \(M : \tau\) in context
\(\Gamma = [\var x_0 : \sigma_0 , \dots , \var x_{k-1} : \sigma_{k-1}]\), and
assume that we have another context \(\Delta\) with terms
\(\Delta \vdash N_i : \sigma_i\) for every \(0 \leq i \leq k-1\).
%
This yields (by a recursive definition on the structure of derivations of
\(\Gamma \vdash M : \tau\)) a term
\[
  \Delta \vdash M[N_0/\var x_0,\dots,N_{k-1}/\var x_{k-1}] : \tau
\]
in context \(\Delta\).

Thus, in our interpretation, we get
\begin{equation*}\label{subst-1}\tag{\(\dagger\)}
  \densem{M[N_0/\var x_0,\dots,N_{k-1}/\var x_{k-1}]}(\delta) \in \densem{\tau}
\end{equation*}
for every \(\delta \in \densem{\Delta}\).

Alternatively, we can note that \(\densem{N_i}(\delta) \in \densem{\sigma_i}\)
for every \(0 \leq i \leq k-1\), so that we can feed them as inputs to
\(\densem{M} \colon \densem{\sigma_0} \times \cdots \times \densem{\sigma_{k-1}}
\to \densem{\tau}\) which yields:
\begin{equation*}\label{subst-2}\tag{\(\ddagger\)}
  \densem{M}\pa*{\densem{N_0}(\delta),\dots,\densem{N_{k-1}}(\delta)} \in \densem{\tau}.
\end{equation*}

The substitution lemma says that \eqref{subst-1} and \eqref{subst-2} agree.

\begin{lemma}[Substitution lemma]\label{substitution-lemma}
  If
  \([\var x_0 : \sigma_0,\dots,\var x_{k-1} : \sigma_{k-1}] = \Gamma \vdash M :
  \tau\), then for every context \(\Delta\) and terms
  \(\Delta \vdash N_i : \sigma_i\) with \(0 \leq i \leq k-1\), we have
  \[
    \densem{M[N_0/\var x_0,\dots,N_{k-1}/\var x_{k-1}]}(\delta)
    =
    \densem{M}\pa*{\densem{N_0}(\delta),\dots,\densem{N_{k-1}}(\delta)}
  \]
  for every \(\delta \in \densem{\Delta}\).
\end{lemma}
\begin{proof}
  By induction on the structure of derivations of \(\Gamma \vdash M : \tau\).

  For example, for the case of \(\lambda\)-abstraction, we consider the term
  \(\Gamma,y : \tau \vdash M : \rho\) with
  \([\var x_0 : \sigma_0,\dots,\var x_{k-1} : \sigma_{k-1}] = \Gamma\), and we
  assume to have a context \(\Delta\) with terms
  \(\Delta \vdash N_i : \sigma_i\) for \(0 \leq i \leq k-1\).
  We have to prove that
  \[
    \densem*{\pa*{\lambdadot{\var y : \tau}{M}}[N_0/\var x_0 , \dots ,
      N_{k-1}/\var x_{k-1}]}(\delta) = \densem*{\lambdadot{\var y :
        \tau}{M}}\pa*{\densem{N_0}(\delta),\dots,\densem{N_{k-1}}(\delta)}
  \]
  for every \(\delta \in \densem{\Delta}\).

  Note that this is an equality of elements of
  \(\densem{\rho}^{\densem{\tau}}\), i.e.\ an equality of
  (\(\omega\)-continuous) functions, so let \(t \in \densem{\tau}\) be
  arbitrary and note that
  \begin{align*}
    &\hspace{13.5pt} \pa*{\densem*{\pa*{\lambdadot{\var y : \tau}{M}}[N_0/\var x_0 , \dots ,
    N_{k-1}/\var x_{k-1}]}(\delta)}(t) \\
    &= \densem*{M[N_0/\var x_0,\dots,N_{k-1}/\var x_{k-1},\var y/\var y]}(\delta,t) \\
    &= \densem*{M}\pa*{\densem{N_0}(\delta,t),\dots,\densem{N_{k-1}}(\delta,t),\densem{\var y}(\delta,t)}
    &\text{(by IH applied to \(\Gamma,\var y : \tau \vdash M : \rho\))} \\
    &= \densem*{M}\pa*{\densem{N_0}(\delta,t),\dots,\densem{N_{k-1}}(\delta,t),t} \\
    &= \pa*{\densem*{\lambdadot{\var y :
        \tau}{M}}\pa*{\densem{N_0}(\delta),\dots,\densem{N_{k-1}}(\delta)}}(t),
  \end{align*}
  as desired.
  %
  Note that the \(N_i\) in lines 2--4 refer to the terms \(N_i\) in the extended
  context \(\Delta,\var y : \tau\).
\end{proof}

\section{Soundness}\label{sec:soundness}

The soundness theorem expresses that the denotational semantics, in the form of
the Scott model of PCF, respects the operational semantics.

\begin{theorem}[Soundness]\label{soundness}
  For terms \(M\) and \(N\) of the same type in the same context, we have: if
  \(M \bigstep N\), then \(\densem{M} = \densem{N}\).
\end{theorem}
\begin{proof}
  By induction on the structure of the derivations of \(M \bigstep N\).
  %
  We work out two cases: the fixed point operator and application.

  \begin{itemize}
  \item Recall that the former is the rule
  \[
    \AxiomC{\(M(\pcffix_{\sigma} \, M) \bigstep V\)}
    \UnaryInfC{\({\pcffix_{\sigma} \, M} \bigstep V\)}
    \DisplayProof
  \]
  so that we may assume \(\densem{M(\pcffix_\sigma\,M)} = \densem{V}\) and we
  have to prove:
  \[
    \densem{\pcffix_{\sigma} \, M} = \densem{V}.
  \]
  But this holds: if \(M\) is a term in context \(\Gamma\), then for
  every \(\gamma \in \densem{\Gamma}\), we have
  \begin{align*}
    \densem{\pcffix_{\sigma} \,M}(\gamma)
    &= \mu\pa*{\densem{M}(\gamma)}
    &\text{(by definition)} \\
    &= \pa*{\densem{M}(\gamma)}\pa*{\mu\pa*{\densem{M}(\gamma)}}
    &\text{(because \(\mu\) gives a fixed point)} \\
    &= \densem{M\pa*{\pcffix_{\sigma}\,M}}(\gamma)
    &\text{(by definition)} \\
    &= \densem{V}(\gamma) &\text{(by assumption)}.
  \end{align*}
  \item   The big-step rule for application is
  \[
    \AxiomC{\(M \bigstep \lambdadot{\var x : \sigma}{E}\)}
    \AxiomC{\(E[N/x] \bigstep V\)}
    \BinaryInfC{\(M \, N \bigstep V\)}
    \DisplayProof
  \]
  so that we may assume
  \(\densem{M} = \densem{\lambdadot{\var x : \sigma}{E}}\)
  and
  \(\densem{E[N/x]} = \densem{V}\),
  and we have to prove:
  \[
    \densem{M \, N} = \densem{V}.
  \]
  If \(M\) and \(N\) are terms in a context \(\Gamma\), then for every
  \(\gamma \in \densem{\Gamma}\), we calculate:
  \begin{align*}
    \densem{M \, N}(\gamma)
    &= \densem{M}(\gamma)\pa*{\densem{N}(\gamma)}
    &\text{(by definition)} \\
    &= \densem{\lambdadot{\var x : \sigma}{E}}(\gamma)\pa*{\densem{N}(\gamma)}
    &\text{(by assumption on \(\densem{M}\))} \\
    &= \densem*{\pa*{\lambdadot{\var x : \sigma}{E}}N}(\gamma)
    &\text{(by definition)} \\
    &= \densem{E[N/\var x]}(\gamma)
    &\text{(by \cref{beta-equality})} \\
    &= \densem{V}(\gamma)
    &\text{(by assumption on \(\densem{N}\))},
  \end{align*}
  completing the proof. \qedhere
  \end{itemize}
\end{proof}

\section{List of exercises}
\begin{enumerate}
\item \cref{exer:interpretation-of-numerals}: On the interpretation of the numerals.
\item \cref{exer:beta-equality}: On proving that the interpretation validates
  \(\beta\)-equality.
\end{enumerate}

%%% Local Variables:
%%% mode: latexmk
%%% TeX-master: "../main"
%%% End:

\chapter{Computational adequacy}\label{chap:comp-adequacy}

Soundness is a nice and fundamental result, but a converse would be more
interesting: Can we use the model to compute in PCF? More precisely, if two
terms \(M\) and \(N\) are equal in the model, does \(M\) reduce to \(N\) (or
vice versa) in the operational semantics?

This is actually not the case, because the model is more extensional than PCF,
e.g.\ the programs \(\lambdadot{\var x : \pcfnat}{\var x}\) and
\(\lambdadot{\var x : \pcfnat}{\pcfpred\pa*{\pcfsuc\,\var x}}\) are different
values in PCF, but their interpretations as \(\omega\)-continuous functions are
equal.

However, it \emph{is} the case that if \(M\) is a program of type \(\pcfnat\)
and \(\densem{M} = n\), then \(M \bigstep \numeral{n}\).
%
This is known as the \emph{computational adequacy} of the Scott model of PCF and
allows us to compute in PCF using the domain-theoretic denotational semantics.

\begin{theorem*}[Computational adequacy]\label{adequacy}
  For every program \(M\) of type \(\pcfnat\) and natural number \(n \in \Nat\),
  if \(\densem{M} = n\), then \(M \bigstep \numeral n\).
\end{theorem*}

Unlike soundness, computational adequacy cannot be proved by a straightforward
induction on types, or structure of the derivation of the program \(M\), since
the statement refers to closed terms of type \(\pcfnat\) only.

Therefore, instead of proving computational adequacy directly, we will derive it
as a corollary of a more general result that does allow for a proof by
induction on types.

More specifically, we will introduce a \emph{logical
  relation}~\cite{Plotkin1973}. It is an example of a fundamental and often used
technique in the theory of programming languages, going back to \emph{Tait's
  method of computability}~\cite{Tait1967} and also known as the method of
\emph{reducibility candidates}~\cite{Girard1989}.

\section{The logical relation}

We introduce the logical relation \(R_\sigma\), a binary relation between
semantics and syntax of PCF, and use it to prove computational adequacy.

\begin{definition}[Logical relation \(\logrel_\sigma\)]\label{def:logical-relation}
  We define a binary relation \(\logrel_\sigma\) between elements of
  \(\densem{\sigma}\) and programs of type \(\sigma\) by induction on types:
  \begin{enumerate}[(i)]
  \item\label{R-nat} \({x \logrel_{\pcfnat} M}\) holds if \(x = n\) with
    \(n \in \Nat\) implies \(M \bigstep \numeral n\),
  \item\label{R-fun} \(f \logrel_{\sigma \pcffun \tau} M\) holds if \(x \logrel_\sigma N\)
    implies \({f(x)} \logrel_\tau {M \, N}\) for every element
    \(x \in \densem{\sigma}\) and program \(N\) of type \(\sigma\).
  \end{enumerate}
\end{definition}

Observe that computational adequacy is equivalent to the statement that for
every program \(M : \pcfnat\) we have \(\densem{M} \logrel_\pcfnat M\).
%
Hence, we will have proven computational adequacy if we can show:

\begin{lemma*}
  For every program \(M : \sigma\), it holds that \(\densem{M} \logrel_\sigma M\).
\end{lemma*}

This, in turn, will follow from the fundamental theorem of the logical relation:

\begin{lemma*}[Fundamental theorem of the logical relation]\label{fundamental-theorem}
  If
  \([\var x_0 : \sigma_0 , \dots , \var x_{k-1} : \sigma_{k-1}] = \Gamma \vdash
  M : \tau\), then whenever we have \(\Delta \vdash N_i : \sigma_i\) and
  \(s_i \in \densem{\sigma_i}\) such that \(s_i \logrel_{\sigma_i} N_i\) for
  all \(0 \leq i \leq k-1\), it holds that
  \[
    \pa*{\densem{M}\pa*{s_0,\dots,s_{k-1}}} \logrel_{\tau} {M[N_0/\var x_0,\dots,N_{k-1}/\var x_{k-1}]}.
  \]
\end{lemma*}

The second clause, item~\ref{R-fun}, of \cref{def:logical-relation} says that
related inputs must go to related outputs and is designed to make proofs by
induction on types possible.

Before we can prove the fundamental theorem of the logical relation, we need to
establish several properties of the relation \(\logrel_\sigma\).

\begin{lemma}\label{R-extend-left}
  If \(x \below y\) and \(y \logrel_\sigma M\), then \(x \logrel_\sigma M\).
\end{lemma}
\begin{proof}
  We proceed by induction on the type \(\sigma\).
  %
  For the base case, suppose we have \(x \below y\) in \(\Nat_\bot\) and
  \(y \logrel_\pcfnat M\), and assume that \(x = n\) for a natural number
  \(n\). We have to prove that \(M \bigstep \numeral n\).
  %
  Since \(n = x \below y\), we must have \(y = n\) by definition of the partial
  order on \(\Nat_\bot\), and hence \(M \bigstep \numeral n\) because we assumed
  \(y \logrel_\pcfnat M\).

  For function types, we assume to have \(f \below g\) in
  \(\densem{\tau}^{\densem{\sigma}}\) with
  \(g \logrel_{\sigma \pcffun \tau} M\), and we have to prove
  \(f \logrel_{\sigma \pcffun \tau} M\).
  %
  So suppose that we have \(x \logrel_\sigma N\). Then
  \(g(x) \logrel_\tau M\,N\) because we assumed
  \(g \logrel_{\sigma \pcffun \tau} M\).
  %
  But \(f \below g\), so \(f(x) \below g(x)\) and hence, we get the desired
  \(f(x) \logrel_\tau M\,N\) by the induction hypothesis applied at the type
  \(\tau\).
\end{proof}

\begin{lemma}\label{contains-bot}
  The least element is related to all programs.
  %
  That is, for every program \(M : \sigma\), we have \(\bot \logrel_\sigma M\),
  where we recall that \(\bot\) denotes the least element of
  \(\densem{\sigma}\).
\end{lemma}
\begin{lemma}\label{closure-under-omega-chains}
  The logical relation is closed under least upper bounds of \(\omega\)-chains.
  %
  That is, for every program \(M : \sigma\) and \(\omega\)-chain
  \(x_0 \below x_1 \dots\) in \(\densem{\sigma}\), if \(x_n \logrel_\sigma M\)
  for every \(n \in \Nat\), then \(\pa*{\bigsqcup_{n \in \Nat}x_n} \logrel_\sigma M\).
\end{lemma}

\begin{exercise}\label{exer:contains-bot-and-closure-under-omega-chains}
  Prove~\cref{contains-bot,closure-under-omega-chains} by induction on PCF
  types.
\end{exercise}

\begin{exercise}\label{exer:closure-under-basic-operations}
  Check the following closure properties of \(\logrel_\pcfnat\):
  \begin{enumerate}[(i)]
  \item \(0 \logrel_\pcfnat \numeral{0}\);
  \item if \(x \logrel_\pcfnat M\), then \(s(x) \logrel_\pcfnat \pcfsuc\, M\),
    where \(s\) is the successor map on \(\Nat_\bot\) as
    in~\cref{def:interpretation}\ref{def:interpretation-succ};
  \item if \(x \logrel_\pcfnat M\), then \(p(x) \logrel_\pcfnat \pcfpred\, M\),
    where \(p\) is the predecessor map on \(\Nat_\bot\) as
    in~\cref{def:interpretation}\ref{def:interpretation-pred};
  \item if \(x \logrel_\pcfnat M\), \(y \logrel_\pcfnat N_1\) and
    \(z \logrel_\pcfnat N_2\), then
    \(c(x,y,z) \logrel_\pcfnat \pcfifz(M,N_1,N_2)\), where \(c\) is the if-zero
    map on \(\Nat_\bot\) as
    in~\cref{def:interpretation}\ref{def:interpretation-ifzero}.
  \end{enumerate}
\end{exercise}

\section{Applicative approximation}

We will need one more ingredient for proving the fundamental theorem of the
logical relation and (hence) computational adequacy: the \emph{applicative
  approximation} preorder.

In~\cref{R-extend-left}, we showed that if \(x \below y\) and
\(y \logrel_\sigma M\), then \(x \logrel_\sigma M\).
%
The applicative approximation may be motivated by the desire to have a similar
property of the logical relation, but for programs, rather than elements of the
interpretation. That is, we introduce a binary relation \(\appapprox_\sigma\) on
programs of type \(\sigma\) and show (\cref{R-extend-right}) that if
\(x \logrel_\sigma M\) and \(M \appapprox_\sigma N\), then
\(x \logrel_\sigma N\).

\begin{definition}[Applicative approximation]
  We define a binary relation \(\appapprox_\sigma\) on programs of type
  \(\sigma\) by induction on types:
  \begin{enumerate}[(i)]
  \item \(M \appapprox_\pcfnat N\) holds if \(M \bigstep \numeral{n}\) implies
    \(N \bigstep \numeral{n}\) for every natural number \(n\in\Nat\).
  \item \(M \appapprox_{\sigma \pcffun \tau} N\) holds if for every program
    \(K\) of type \(\sigma\) we have \(M \, K \appapprox_\tau N \, K\).
  \end{enumerate}
\end{definition}

Note that the applicative approximation relation mirrors the partial orders on
\(\Nat_\bot\) and the exponentials.
%
But the applicative approximation relation is only a \emph{preorder}; it is
reflexive and transitive, but not antisymmetric:

\begin{exercise}\label{exer:applicative-approximation-not-antisymmetric}
  Give examples of programs \(M\) and \(N\) such that
  \(M \appapprox_{\pcfnat \pcffun \pcfnat} N\) and
  \(N \appapprox_{\pcfnat \pcffun \pcfnat} M\), but \(M \neq N\).
\end{exercise}

\begin{lemma}\label{R-extend-right}
  If \(x \logrel_\sigma M\) and \(M \appapprox_\sigma N\), then
  \(x \logrel_\sigma N\).
\end{lemma}
\begin{proof}
  We proceed by induction on types.
  %
  For the base type, we assume \(x \logrel_\pcfnat M\) and
  \(M \appapprox_\pcfnat N\). Suppose that \(x = n\) for \(n \in \Nat\); we have
  to prove \(N \bigstep \numeral n\). But \(M \bigstep \numeral n\) since
  \(x \logrel_\pcfnat M\) and hence, \(N \bigstep \numeral n\) because
  \(M \appapprox_\pcfnat N\).

  For function types, we assume \(f \logrel_{\sigma \pcffun \tau} M\) and
  \(M \appapprox_{\sigma \pcffun \tau} N\). Suppose that we have
  \(x \logrel_\sigma K\); we have to prove \(f(x) \logrel_\tau N \, K\).
\end{proof}

The following lemma has two useful corollaries for proving the fundamental
theorem of the logical relation.
\begin{lemma}\label{applicative-approximation-if-big-step-implication}
  For all programs \(M\) and \(N\) of type \(\sigma\), if \(M \bigstep V\)
  implies \(N \bigstep V\) for every program \(V : \sigma\), then
  \(M \appapprox_\sigma N\).
\end{lemma}

\begin{exercise}\label{exer:applicative-approximation-if-big-step-implication}
  Prove~\cref{applicative-approximation-if-big-step-implication} by induction on
  types.
\end{exercise}



\begin{corollary}

\end{corollary}

\begin{corollary}

\end{corollary}


\section{Proving computational adequacy}

Finally, we have established enough properties to prove:
\begin{lemma}[Fundamental theorem of the logical relation]\label{fundamental-theorem}
  If \(M : \tau\) is a term in a context
  \(\Gamma = [\var x_0 : \sigma_0 , \dots , \var x_{k-1} : \sigma_{k-1}]\), then
  whenever we have \(\Delta \vdash N_i : \sigma_i\) and
  \(s_i \in \densem{\sigma_i}\) such that \(s_i \logrel_{\sigma_i} N_i\) for all
  \(0 \leq i \leq k-1\), it holds that
  \[
    \pa*{\densem{M}\pa*{s_0,\dots,s_{k-1}}} \logrel_{\tau} {M[N_0/\var x_0,\dots,N_{k-1}/\var x_{k-1}]}.
  \]
\end{lemma}
\begin{proof}
  We proceed by induction on the structure of the derivations
  \(\Gamma \vdash M : \tau\):

  \begin{itemize}
  \item For
    \([\var x_0 : \sigma_0,\dots,\var x_{k-1} : \sigma_{k-1}] \vdash \var x_i :
    \sigma_i\) and \(s_i \logrel_{\sigma_i} N_i\) for all
    \(0 \leq i \leq k-1\), we observe that
    \[ {\densem{\var x_i}(s_0,\dots,s_{k-1})} = {s_i} \logrel_{\sigma_i} {N_i} =
      {{\var x_i}[N_0/\var x_0,\dots,N_{k-1}/\var x_{k-1}]},
    \]
    as wished.
  \item For \(\lambda\)-abstraction, we ...
  \item For application, we ...
  \item
    For \(\pcffix\), we ...
  \item Finally, the cases for \(\pcfzero\), \(\pcfsuc\), \(\pcfzero\) and
    \(\pcfifz\) follow from~\cref{exer:closure-under-basic-operations}. \qedhere
  \end{itemize}
\end{proof}

For the special case of the empty context, we obtain:
\begin{corollary}
  For every program \(M : \sigma\), it holds that
  \(\densem{M} \logrel_\sigma M\).
\end{corollary}
%
By further specialising to the base type and the definition of
\(\logrel_\pcfnat\), we get:
\begin{theorem}[Computational adequacy]\label{adequacy}
  For every program \(M\) of type \(\pcfnat\) and natural number \(n \in \Nat\),
  if \(\densem{M} = n\), then \(M \bigstep \numeral n\).
\end{theorem}


\section{List of exercises}
\begin{enumerate}
\item \cref{exer:contains-bot-and-closure-under-omega-chains}: On proving that
  the logical relation contains the least element is and is closed under least
  upper bounds of \(\omega\)-chains.
\item \cref{exer:closure-under-basic-operations}: On showing that the logical
  relation is suitably closed under the basic operations on the type of natural
  numbers.
\item \cref{exer:applicative-approximation-not-antisymmetric}: On a
  counterexample to antisymmetry of the applicative approximation relation.
\item \cref{exer:applicative-approximation-if-big-step-implication}: On a
  sufficient big-step condition for the applicative approximation relation to
  hold.
\end{enumerate}

%%% Local Variables:
%%% mode: latexmk
%%% TeX-master: "../main"
%%% End:


\backmatter%
\printbibliography[heading=bibintoc]%

\end{document}
