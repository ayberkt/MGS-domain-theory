\chapter[{Denotational semantics and domain theory}]{Denotational semantics and \\ domain theory}\label{chap:domains}

Previously we introduced the operational semantics of PCF. Next, we wish to
introduce a \emph{denotational semantics} to PCF. The basic idea is to assign to
each type \(\sigma\) of PCF some kind of \emph{mathematical} object
\(\densem{\sigma}\).
%
A program \(M\) (i.e.\ closed terms) of type \(\sigma\) is then interpreted as
an element \(\densem{M}\) of \(\densem{\sigma}\).
%
More generally, we extend the interpretation of types to contexts and
\(\Gamma \vdash M : \sigma\) will then be interpreted as a certain
(\(\omega\)-continuous) function from \(\densem{\Gamma}\) to
\(\densem{\sigma}\).

For the denotational semantics we have three desiderata:

\begin{itemize}[itemsep=2mm,topsep=3mm]
\item\emph{Compositionality}: This can be summarised as follows:
  \begin{displayquote}
    \emph{The interpretation of a composite is the composite of the interpretations}.
  \end{displayquote}

  %
  For example, the interpretation of a function type \(\sigma \pcffun \tau\)
  will be a certain set of functions from \(\densem{\sigma}\) to
  \(\densem{\tau}\), and if we have programs \(M : \sigma \pcffun \tau\) and
  \(N : \sigma\), then \(\densem{M \, N} = \densem{M}(\densem{N})\).
\item\emph{Soundness}: We want our interpretation to respect the operational
  semantics, i.e.\ if we have terms \(M\) and \(N\) with
  \(M \bigstep N\), then their interpretations should be equal.
\item\emph{Computational adequacy}: We should be able to use our
  interpretation to compute, i.e.\ if \(M\) is a program of type \(\pcfnat\)
  and \(\densem{M} = n\) for some natural number \(n\), then
  \(M \bigstep \numeral{n}\).
\end{itemize}

Soundness and computational adequacy will be proved
in~\cref{sec:soundness,chap:comp-adequacy}, while compositionality will be a
direct consequence of our definitions.

\section{Towards domain theory}

All this leaves the question what kind of mathematical objects we have in mind
for interpreting the types of PCF. A first, naive attempt may be to interpret
each type \(\sigma\) as a set \(\densem{\sigma}\) with:
\begin{align*}
  \densem{\pcfnat} &\coloneqq \text{the set \(\Nat\) of natural numbers}, \text{ and} \\
  \densem{\sigma \pcffun \tau} &\coloneqq \text{the set \(\set{f \colon \densem{\sigma} \to
    \densem{\tau}}\) of all functions from \(\densem{\sigma}\) to \(\densem{\tau}\)}.
\end{align*}

There are two problems with this naive approach.
%
The first problem is that \(\densem{\pcfnat}\) should not only offer natural
numbers, but it should also offer an interpretation of programs of type
\(\pcfnat\) that do not terminate such as the one
in~\cref{exam:non-termination}.
%
Right now, this program \(M\) would have to be interpreted as some natural
number \(n\), forcing us to make some arbitrary choice. Moreover, it is not true
that \(M \bigstep \numeral n\), whatever choice we make, so we would violate
computational adequacy.

This problem is easily solved by introducing a new, special element
\(\bot_\sigma \in \densem{\sigma}\) that interprets non-terminating programs
of type \(\sigma\). And indeed, this will be a part of our eventual interpretation.

However, it does not solve the second, more serious problem, namely that, in
order to interpret PCF's \(\pcffix_\sigma\) we would need every function
\(f \colon \densem{\sigma} \to \densem{\sigma}\) to have a fixed point.
%
But this is easily seen to \emph{false}, e.g. consider
\begin{align*}
  f \colon \Nat \cup \{\bot\} &\to \Nat \cup \{\bot\} \\
  \bot &\mapsto 0, \\
  n &\mapsto n+1.
\end{align*}

This is where domain theory comes to the rescue: Instead of using plain sets, we
will interpret types as so-called \emph{\(\omega\)-cppos}: these are
\emph{partially ordered} sets with a \emph{least} element \(\bot\) and
\emph{least upper bounds} of increasing sequences.
%
A function between (the underlying sets of) two \(\omega\)-cppos is
\emph{\(\omega\)-continuous} if it preserves the order and these least upper
bounds.
%
The upshot of restricting to such \(\omega\)-continuous functions is that these
can be shown to have (least) fixed points, thus solving our second problem.

\section{Basic definitions and the least fixed point theorem}

We define \(\omega\)-cppos (short for: pointed,
\(\omega\)-chain complete posets) and \(\omega\)\nobreakdash-continuous
functions between them.

\begin{definition}[Poset]
  A \emph{partially ordered set}, or \emph{poset}, is a set \(X\) together with
  a binary relation \({\below}\) satisfying:
  \begin{enumerate}[(i)]
  \item \emph{reflexivity}: for every \(x \in X\), we have \(x \below x\);
  \item \emph{transitivity}: for every \(x,y,z \in X\), if \(x \below y\) and
    \(y \below z\), then \(x \below z\);
  \item \emph{antisymmetry}: for every \(x,y \in X\), if \(x \below y\) and
    \(y \below x\), then \(x = y\). \qedhere
  \end{enumerate}
\end{definition}

\begin{example}
  The natural numbers, rational numbers and real numbers with their usual
  orderings are all examples of posets.
\end{example}

\begin{definition}[\(\omega\)-chain, least upper bound, \(\omega\)-cpo]
  Let \((X,{\below})\) be a poset.
  \begin{enumerate}[(i)]
  \item An \(\omega\)-chain in \(X\) is a subset
    \(\set{x_0,x_1,x_2,\dots} \subseteq X\) of increasing elements, i.e.\ we
    have \(x_0 \below x_1\), \(x_1 \below x_2\), \(x_2 \below x_3\), and so
    on.% \below \dots\).
  \item An \emph{upper bound} of a subset \(S \subseteq X\) is an element
    \(x \in X\) such that \(s \below x\) for every \(s \in S\).
  \item A \emph{least upper bound} of a subset \(S \subseteq X\) is an upper
    bound \(x \in X\) of \(S\) such that for every upper bound \(y \in X\) of
    \(S\) we have \(x \below y\).
  \item The poset \(X\) is \emph{\(\omega\)-chain complete} if every
    \(\omega\)-chain in \(X\) has a least upper bound.% in \(X\).
  \end{enumerate}
  An \(\omega\)-chain complete poset will be called an \emph{\(\omega\)-cpo}.
\end{definition}

\begin{exercise}\label{exer:least-upper-bounds-are-unique}
  Show that least upper bounds are unique. That is, if \(x\) and \(y\) are least
  upper bounds of some subset \(S\) of a poset \(X\), then \(x = y\).

  \emph{Hint}: Use antisymmetry.
\end{exercise}

Because least upper bounds are unique, we will speak of \emph{the} least upper
bound (of a given subset) and in the case of \(\omega\)-chain
\(x_0 \below x_1 \dots\), we will denote the least upper bound by
\(\bigsqcup_{n \in \Nat}x_n\).

\begin{definition}[Least element, \(\omega\)-cppo]
  Let \((X,{\below})\) be a poset.
  \begin{enumerate}[(i)]
  \item A \emph{least element} of \(X\) is an element \(x \in X\) such that
    \(x \below y\) for every \(y \in X\).
  \item An \(\omega\)-cpo is called \emph{pointed} if it has a least element.
  \end{enumerate}
  A pointed \(\omega\)-cpo will be called an \emph{\(\omega\)-cppo}.
\end{definition}

\begin{exercise}\label{exer:least-element-is-unique}
  Let \(X\) be a poset with a least element \(x\). Show that \(x\) is the least
  upper bound of the empty subset and hence conclude that \(x\) must be unique.
\end{exercise}

Because least elements are unique, we will speak of \emph{the} least element and
will typically denote it by \(\bot\).

\begin{example}
  The set of natural numbers \(\Nat\) with the usual order is \emph{not} an
  \(\omega\)-cpo, because the \(\omega\)-chain
  \(\set{0,1,\dots}\) does not have a (least) upper bound.
\end{example}

\begin{example}[\(\Nat_\bot\)]\label{exam:N_bot}
  A fundamental example of an \(\omega\)-cppo is the \emph{flat} \(\omega\)-cppo
  \(\Nat_\bot\), which is given by the set \(\Nat \cup \{\bot\}\) ordered as
  depicted in the following diagram:
  \[
    \begin{tikzcd}
      0 & 1 & 2 & 3 & \ldots \\
      & & \bot \ar[ull,no head] \ar[ul,no head] \ar[u,no head] \ar[ur, no head]
      \ar[urr, no head]
    \end{tikzcd}
  \]
  Here we interpret a line going up from an element \(x\) to an element \(y\) as
  saying that \(x \below y\).
  %
  Thus, the element \(\bot\) is the least element and all other elements are
  unrelated.
  %
  In particular, we have do \emph{not} have \(0 \below 1\) as in the usual
  ordering of the natural numbers.

  The intuition here is that the partial order \({\below}\) does not reflect the
  numerical value, but rather how ``defined'' an element is with \(\bot\)
  representing an ``undefined''.
  %
  In our interpretation of PCF, we will interpret \(\pcfnat\) as \(\Nat_\bot\)
  and \(\bot\) will serve as an interpretation of non-terminating programs.
\end{example}

\begin{exercise}\label{exer:N_bot}
  Show that the \(\omega\)-chains in \(\Nat_\bot\) are precisely \(\set{\bot}\),
  \(\set{n}\) and \(\set{\bot,n}\) for a natural number \(n \in \Nat\).
  %
  Use this to check that \(\Nat_\bot\) is indeed an \(\omega\)-cpo.
\end{exercise}

\begin{remark}
  It is perhaps a bit arbitrary that to exhibit
  \(\set{\bot,n} \subseteq \Nat_\bot\) as an \(\omega\)-chain we have to repeat
  elements and order them linearly.
  %
  One way to address this is to replace the notion of \(\omega\)-chain by that
  of a \emph{directed} (see~\cref{exer:directedness}) subset, and to consider
  \emph{dcpos}: posets with least upper bounds for all directed subsets.

  For general domain theory the notion of dcpo is arguably the better notion
  (see also~\cite[Section~2.2.4]{AbramskyJung1994}, but for the denotational
  semantics of PCF and in particular, the existence of least fixed
  points~(\cref{least-fixed-point}), the notion of \(\omega\)-cpo suffices.
\end{remark}

\begin{exercise}\label{exer:directedness}
  Let \((X,{\below})\) be a poset. A subset \(S \subseteq X\) is \emph{directed}
  if it is non-empty and for every two elements \(x,y \in S\), there exists a
  element \(z \in S\) with \(x \below z\) and \(y \below z\).
  \begin{enumerate}[(i)]
  \item Show that every \(\omega\)-chain is a directed subset.
  \item Conversely, use the axiom of choice to show that every countable,
    directed subset is an \(\omega\)-chain.\qedhere
  \end{enumerate}
\end{exercise}

We proceed by defining the \(\omega\)-continuous maps between (pointed)
\(\omega\)-cpos.
\begin{definition}[\(\omega\)-continuity]\label{def:continuity}
  A function \(f \colon A \to B\) between the underlying sets of two
  \(\omega\)-cpos \(A\) and \(B\) is \emph{\(\omega\)-continuous} if
  \begin{enumerate}[(i)]
  \item\label{monotone} it is \emph{monotone} (or \emph{order preserving}),
    i.e.\ if \(x \below y\) in \(A\), then \(f(x) \below f(y)\) in \(B\);
  \item\label{preserve-lub-of-omega-chains} it \emph{preserves least upper
      bounds of \(\omega\)-chains}, i.e.\ if \(\set{x_0,x_1,\dots}\) is an
    \(\omega\)-chain in \(A\) with least upper bound \(a\), then \(f(a)\) is the
    least upper bound in \(B\) of the subset \(\set{f(x_0),f(x_1),\dots}\). \qedhere
  \end{enumerate}
\end{definition}

\begin{remark}
  It follows from monotonicity that \(\set{f(x_0),f(x_1),\dots}\) is an
  \(\omega\)-chain in \(B\) and hence we could rewrite the second condition as:
  \[
    f\pa*{\textstyle\bigsqcup_{n \in \Nat}x_n} = \textstyle\bigsqcup_{n \in
      \Nat}f(x_n). \qedhere
  \]
\end{remark}

It is worthwhile to reflect on the computational intuitions underlying
monotonicity and preservation of least upper bounds of \(\omega\)-chains.
%
If we think of \(f \colon A \to B\) as some computational procedure, then
monotonicity says:
\begin{displayquote}
  \emph{more (or better) input leads to more (or better) output},
\end{displayquote}
while the second condition of \(\omega\)-continuity says:
\begin{displayquote}
  \emph{every output can be patched together from the outputs of approximations}.
\end{displayquote}
The idea here is that a computational procedure can only inspect a finite amount
of input before returning an output, so that the approximations suffice to
determine the output.
%
These intuitions and ideas are nicely illustrated in the following example:
\begin{exercise}\label{exer:cantor-domain}
  Let \(C\) be the poset of finite and infinite binary sequences ordered by
  prefix. For an infinite sequence \(\alpha\) we write
  \(\alpha \upharpoonright n\) for its finite prefix of length \(n\).

  \begin{enumerate}[(i)]
  \item Show that \(C\) is an \(\omega\)-cppo.
  \item Show that a function \(f \colon C \to C\) is \(\omega\)-continuous if
    and only if for every infinite sequence \(\alpha\) we have
    \(f(\alpha) = \textstyle\bigsqcup_{n \in \Nat}f(\alpha \upharpoonright n)\). \qedhere
  \end{enumerate}
\end{exercise}

% or alternatively, if we think of \(x \below y\) as expressing that \(y\) is at
% least as informative as \(x\), then \(f(y)\) is at least as informative as
% \(f(x)\).

Actually, monotonicity, item~\ref{monotone} in~\cref{def:continuity}, is
redundant as the following exercise ask you to check:

\begin{exercise}\label{exer:monotonicity-follows}
  Show that if a function \(f\) between \(\omega\)-cpos satisfies
  item~\ref{preserve-lub-of-omega-chains} of~\cref{def:continuity}, then \(f\)
  must be monotone.

  \emph{Hint}: If \(x \below y\), then \(\set{x,y}\) is an \(\omega\)-chain with
  least upper bound \(y\).
\end{exercise}

\begin{example}
  The function \(f \colon \Nat_\bot \to \Nat_\bot\) given by \(\bot \mapsto 0\)
  and \(n \mapsto n + 1\) is \emph{not} \(\omega\)-continuous, because it is not
  monotone: \(\bot \below 1\), but \(f(\bot) = 0\) is not below \(f(1) = 2\) in
  the ordering of \(\Nat_\bot\).

  On the other hand, the function \(g \colon \Nat_\bot \to \Nat_\bot\) given by
  \(\bot \mapsto \bot\) and \(n \mapsto n+1\) is \(\omega\)-continuous.
  %
  In fact, it will serve as the interpretation of the PCF program
  \(\lambdadot{\var x : \pcfnat}{\pcfsuc\,\var x}\) of type
  \(\pcfnat \pcffun \pcfnat\).
\end{example}

The point of introducing \(\omega\)-continuity is the following fundamental
result:

\begin{theorem}[Knaster--Tarski fixed point theorem]\label{least-fixed-point}
  Every \(\omega\)-continuous function \(f \colon X \to X\) on a pointed
  \(\omega\)-cpo \(X\) has a least fixed point given by the least upper bound of
  the \(\omega\)-chain
  \[
    \bot \below f(\bot) \below f(f(\bot)) \dots.
  \]
\end{theorem}
\begin{proof}[Proof sketch]
  We write \(f^n(\bot)\) for the \(n\)-fold application of \(f\) to the least
  element \(\bot\).
  %
  We first check that \(\set{f^n(\bot) \mid n \in \Nat}\) is indeed an
  \(\omega\)-chain.
  %
  Since \(\bot\) is the least element, we certainly have
  \(\bot \below f(\bot)\). But now we also have \(f(\bot) \below f^2(\bot)\) by
  monotonicity of \(f\). It follows by induction on \(n\) that
  \(f^n(\bot) \below f^{n+1}(\bot)\), as desired.
  %
  Now we calculate:
  \begin{align*}
    f\pa*{\textstyle\bigsqcup_{n \in \Nat}f^n(\bot)}
    &= \bigsqcup_{n \in \Nat}f\pa*{f^n(\bot)}
    &\text{(since \(f\) preserves least upper bounds of \(\omega\)-chains)} \\
    &= \bigsqcup_{n \in \Nat}f^{n+1}(\bot)
    &\text{(by definition)} \\
    &= \bigsqcup_{n \in \Nat}f^{n}(\bot)
    &\text{(by antisymmetry and calculation)}
  \end{align*}
  so we have a fixed point, as claimed. That it is the least follows
  from~\cref{exer:least-fixed-point}.
\end{proof}

\begin{exercise}[Park induction]\label{exer:least-fixed-point}
  Show that for every \(y \in X\) with \(f(y) \below y\), we have
  \(\bigsqcup_{n \in \Nat}f^n(\bot) \below y\).
  %
  Conclude that \(\bigsqcup_{n \in \Nat}f^n(\bot)\) is indeed the \emph{least} fixed point.
\end{exercise}

\begin{exercise}\label{exer:category-of-cpos}
  Show that \(\omega\)-cpos and \(\omega\)-continuous functions form a
  \emph{category}, i.e.\
  \begin{enumerate}
  \item if \(X\) is an \(\omega\)-cpo, then the identity function
    \(x \mapsto x\) on \(X\) is \(\omega\)-continuous, and
  \item if we have \(\omega\)-continuous functions \(f \colon A \to B\) and
    \(g \colon B \to C\), then so is their composition (as functions)
    \(g \circ f \colon A \to C\). \qedhere
  \end{enumerate}
\end{exercise}

\begin{exercise}[For those familiar with category theory]\label{exer:adjunctions}

  Write \(\omegaCPO\) for the category constructed above and \(\omegaCPPObot\)
  for the category of \emph{pointed} \(\omega\)-cpos and those morphisms of
  \(\omega\)-cpos that preserve the least element.

  Construct adjunctions
  \[
  \begin{tikzcd}
    \Set
    \arrow[r, ""{name=F}, bend left=35] &
    \omegaCPO
    \arrow[l, ""{name=G}, bend left=25]
    \arrow[phantom, from=F, to=G, "\dashv" rotate=-90]
    \arrow[r, ""{name=F}, bend left=25] &
    \omegaCPPObot
    \arrow[l, ""{name=G}, bend left=20]
    \arrow[phantom, from=F, to=G, "\dashv" rotate=-90]
  \end{tikzcd}
\]
\emph{Hint}: The composite functor \(\Set \to \omegaCPPObot\) takes the set \(\Nat\) to
the \(\omega\)-cppo \(\Nat_\bot\) from~\cref{exam:N_bot}.
\end{exercise}

\section{Products: interpreting contexts}

As explained in the introduction to this chapter our goal is to interpret the
types of PCF as \(\omega\)-cppos. Thus, we will have an \(\omega\)-cppo
\(\densem{\sigma}\) for each PCF type \(\sigma\).
%
Moreover, we wish to extend this interpretation to contexts, which are lists of
typed variables. In a context
\(\Gamma = [\var x_0 : \sigma_0, \var x_1 : \sigma_1 , \dots , \var x_n :
\sigma_{n-1}]\) the variables themselves are not really important; it's the
types that matter. Accordingly, we will interpret such a context as the
\emph{product}
\(\densem{\sigma_0} \times \densem{\sigma_1} \times \cdots \times
\densem{\sigma_{n-1}}\) of the interpretations of the types.

The empty context will be interpreted as follows:

\begin{example}[The one point \(\omega\)-cppo \(\One\)]\label{exam:one-point}
  The one point \(\omega\)-cppo \(\One\) is the unique \(\omega\)-cppo with the
  singleton \(\set{\star}\) as its underlying set.
\end{example}

\begin{definition}[Binary product of \(\omega\)-cpos, \(A \times B\)]
  Given two \(\omega\)-cpos \(A\) and \(B\), their \emph{binary product}
  \(A \times B\) is defined by taking the cartesian product of their underlying
  sets with pairwise partial order:
  \[
    (x_1 , y_1) \below_{A \times B} (x_2 , y_2) %
    \coloneqq (x_1 \below_A x_2) \text{ and } (y_1 \below_B y_2).
  \]
  Given an \(\omega\)-chain \(\set{(x_0,y_0),(x_1,y_1),\dots}\) in
  \(A \times B\), one verifies that \(\set{x_0,x_1,\dots}\) and
  \(\set{y_0,y_1,\dots}\) are \(\omega\)-chains in \(A\) and \(B\),
  respectively.
  %
  Hence, we can consider their least upper bounds \(\bigsqcup_{n \in \Nat}x_n\)
  and \(\bigsqcup_{n \in \Nat}y_n\) in \(A\) and \(B\).
  %
  One then checks that the least upper bound of the original \(\omega\)-chain in
  \(A \times B\) is given by the pair of least upper bounds
  \(\pa*{\bigsqcup_{n \in \Nat}x_n , \bigsqcup_{n \in \Nat}y_n}\).

  Finally, if \(A\) and \(B\) are pointed, then so is \(A \times B\) with least
  element \(\pa*{\bot_A , \bot_B}\).
\end{definition}

\begin{example}
  The product of the \(\omega\)-cppos %
  \(\begin{tikzcd}[row sep=8pt]
    x \\
    \bot \ar[u, no head]
  \end{tikzcd}\)
  and \( \begin{tikzcd}[row sep=8pt]
    y \\
    \bot \ar[u, no head]
  \end{tikzcd}
  \) looks like this:

  \[
    \begin{tikzcd}[row sep=15pt,column sep=1pt]
      & (x,y) \\
      (x,\bot) \ar[ur, no head]& & (\bot,y) \ar[ul, no head] \\
      & (\bot,\bot) \ar[ur,no head] \ar[ul, no head]
    \end{tikzcd}\qedhere
  \]
\end{example}

\begin{example}[An illustration of the \(\omega\)-cppo \(\Nat_\bot \times \Nat_\bot\)]
  \[
    \begin{tikzcd}[column sep=1pt, row sep=50pt]
      & (0,0) & (0,1) & (1,0) & (1,1) & \dots \\
      (\bot,0) \ar[ur, no head, shift right=5pt]
      \ar[urrr, no head, shift right=5pt]
      %\ar[urrrrr, no head, shift right=5pt]
      & (\bot,1)
      \ar[ur, no head, shift right=5pt]
      \ar[urrr, no head, shift right=5pt]
      %\ar[urrrr, no head, shift right=5pt]
      & \dots
      &
      & (0,\bot)
      \ar[ulll, no head, shift left=5pt]
      \ar[ull, no head, shift left=5pt]
      & (1,\bot)
      \ar[ull, no head, shift left=5pt]
      \ar[ul, no head, shift left=5pt]
      & \dots \\
      & & & (\bot,\bot)
      %
      \ar[ulll,no head, shift left=5pt] \ar[ull, no head, shift left=5pt]
      \ar[ul, no head, shift left=5pt] \ar[ur, no head, shift right=5pt]
      \ar[urr, no head, shift right=5pt] \ar[urrr, no head, shift right=5pt]
    \end{tikzcd}\qedhere
  \]
\end{example}

The construction of least upper bounds of \(\omega\)-chains in the product
ensures:
\begin{lemma}\label{projections}
  Given two \(\omega\)-cpos \(A\) and \(B\), the \emph{projections}
  \begin{align*}
    \pi_1 \colon A \times B &\to A &&& \pi_2 \colon A \times B &\to B \\
    (x,y) &\mapsto x &&& (x,y) &\mapsto y
  \end{align*}
  are \(\omega\)-continuous.
\end{lemma}

\begin{exercise}\label{exer:continuity-in-each-argument}
  Show that a function \(f \colon A \times B \to C\) between \(\omega\)-cpos is
  \(\omega\)-continuous if and only if the functions \(f(x,-) \colon B \to C\) and
  \(f(-,y) \colon A \to C\) are \(\omega\)-continuous for every \(x \in A\) and
  \(y \in B\).

  Thus, \(\omega\)-continuity of \(f\) can be checked in each argument
  separately.
\end{exercise}

\begin{exercise}\label{exer:product-induced}
  Prove that if \(f \colon C \to A\) and \(g \colon C \to B\) are
  \(\omega\)-continuous functions between \(\omega\)-cpos, then so is the
  induced map
  \begin{align*}
    \langle f , g \rangle  \colon C &\to A \times B \\
    x &\mapsto (f(x),g(x)).
  \end{align*}

  For those familiar with category theory: conclude that the category
  \(\omegaCPO\) has finite products.
\end{exercise}

\section{Exponentials: interpreting function types}

To interpret the function type \(\sigma \Rightarrow \tau\) of PCF using
\(\omega\)-cpos, we are going to construct an \(\omega\)-cpo of
\(\omega\)-continuous functions: the \emph{exponential}.

\begin{definition}[Exponential of \(\omega\)-cpos, \(B^A\)]\label{def:exponential}
  Given two \(\omega\)-cpos \(A\) and \(B\), their \emph{exponential} \(B^A\) is
  defined by equipping the set of \(\omega\)-continuous functions from
  \(A\)~to~\(B\) with the \emph{pointwise} order:
  \[
    f \below g
    \coloneqq \forall_{x \in A} \, \pa{f(x) \below_B g(x)}.
  \]
  Given an \(\omega\)-chain \(\set{f_0,f_1,\dots}\) in \(B^A\), one verifies
  that \(\set{f_0(x),f_1(x),\dots}\) is an \(\omega\)-chain in \(B\) for every
  \(x \in A\).
  %
  Hence, for every \(x \in A\), we can consider the least upper bound
  \(\bigsqcup_{n \in \Nat}f_n(x)\) in \(B\).
  %
  One then checks that the function \(x \mapsto \bigsqcup_{n \in \Nat}f_n(x)\)
  is \(\omega\)-continuous and moreover, that it is the least upper bound of the
  original \(\omega\)-chain in \(B^A\).

  Finally, if \(B\) is pointed, then so is \(B^A\) with least
  element \(x \mapsto \bot\).
\end{definition}

We illustrate the exponential with the following example and exercise before
proceeding to develop the required machinery for interpreting PCF using
\(\omega\)-cppos and \(\omega\)-continuous maps.

\begin{example}
  Consider the \(\omega\)-cpo \(\Two \coloneqq
  \begin{tikzcd}[row sep=5pt]
    \top \ar[d,no head] \\ \bot
  \end{tikzcd}\). Writing \([x,y]\) for the function
  \({f \colon \Two \to \Two}\) given by \(f(\bot) \coloneqq x\) and
  \(f(\top) \coloneqq y\), we picture the exponential \(\Two^\Two\) as:
  \[
  \begin{tikzcd}[row sep=10pt]
    {[\top,\top]} \ar[d, no head] \\
    {[\bot,\top]} \ar[d, no head] \\
    {[\bot,\bot]}
  \end{tikzcd}
  \]
  Note that \([\top,\bot]\) is \emph{not} an element of \(\Two^\Two\), because
  it isn't monotone.
\end{example}

\begin{exercise}\label{exer:approximating-successor}
  For each natural number \(k \in \Nat\), define the function
  \(s_k \colon \Nat_\bot \to \Nat_\bot\) by
  \[
    s_k(\bot) \coloneqq \bot
    \quad\text{and}\quad
    s_k(n) \coloneqq
    \begin{cases}
      n+1 &\text{if \(n \leq k\)},
      \\ \bot &\text{else}.
    \end{cases}
  \]
  \begin{enumerate}[(i)]
  \item Show that each \(s_k\) is \(\omega\)-continuous.
  \item Show that \(\set{s_k \mid k \in \Nat}\) is an \(\omega\)-chain in
    \(\Nat_\bot^{\Nat_\bot}\) and its least upper bound is
    \(s \colon \Nat_\bot \to \Nat_\bot\) with \(s(\bot) \coloneqq \bot\) and
    \(s(n) \coloneqq n+1\).
    % \begin{align*}
    %   s \colon \Nat_\bot &\to \Nat_\bot \\
    %   \bot &\mapsto \bot \\
    %   n &\mapsto n+1.
    % \end{align*}
  \end{enumerate}
  Thus, the functions \((s_k)_{k \in \Nat}\) are increasingly better approximations of the
  successor map on \(\Nat_\bot\).
\end{exercise}

\begin{lemma}\label{evaluation-is-continuous}
  Given two \(\omega\)-cpos \(A\) and \(B\), the \emph{evaluation} function
  \begin{align*}
    B^A \times A &\to B \\
    (f,x) &\mapsto f(x)
  \end{align*}
  is \(\omega\)-continuous.
\end{lemma}
\begin{exercise}\label{exer:evaluation-is-continuous}
  Prove~\cref{evaluation-is-continuous}.

  \emph{Hint}: Use~\cref{exer:continuity-in-each-argument}.
\end{exercise}

\begin{exercise}\label{exer:curry-is-continuous}
  Show that if \(f \colon C \times A \to B\) is an \(\omega\)-continuous
  function between \(\omega\)-cpos, then so is the \emph{curried} function
  \begin{align*}
    \hat{f} \colon C &\to B^A \\
    \hat{f}(c) &\coloneqq a \mapsto f(c,a).
  \end{align*}

  For those familiar with category theory: conclude that the category
  \(\omegaCPO\) is cartesian closed.
\end{exercise}

The following theorem says that the construction of the Knaster--Tarski least
fixed point (\cref{least-fixed-point}) is continuous and will be used to
interpret PCF's fixed point operator
\(\pcffix_\sigma : \pa*{\sigma \pcffun \sigma} \pcffun \sigma\).
\begin{theorem}\label{least-fixed-point-is-continuous}
  For every \(\omega\)-cpo \(A\), the assignment
  \begin{align*}
    \mu \colon A^A &\to A \\
    f &\mapsto \textstyle\bigsqcup_{n \in \Nat}f^n(\bot)
  \end{align*}
  of an \(\omega\)-continuous endofunction on \(A\) to its least fixed point is
  \(\omega\)-continuous.
\end{theorem}
\begin{proof}[Proof sketch]
  For each natural number \(n \in \Nat\), define
  \begin{align*}
    \iter_n \colon A^A &\to A \\
    f &\mapsto f^n(\bot).
  \end{align*}
  By induction on \(n\), we see that each \(\iter_n\) is \(\omega\)-continuous.
  %
  Thus, they are elements of the exponential \(A^{\pa{A^A}}\).
  %
  Moreover, \(\set{\iter_n \mid n \in \Nat}\) is an \(\omega\)-chain in this
  \(\omega\)-cpo.
  %
  Hence, we can take its least fixed point \(F \in A^{\pa{A^A}}\) which is an
  \(\omega\)-continuous function by construction.
  %
  Finally, from the construction of least upper bounds in the exponential, we
  calculate that \(F(f) = \mu(f)\) for every \(f \in A^A\), so that \(\mu\)
  must be \(\omega\)-continuous.
\end{proof}

\section{List of exercises}
\begin{enumerate}
\item \cref{exer:least-upper-bounds-are-unique}: On the uniqueness of least upper bounds.
\item \cref{exer:least-element-is-unique}: On the uniqueness of least elements.
\item \cref{exer:N_bot}: On \(\omega\)-chains in \(\Nat_\bot\).
\item \cref{exer:directedness}: On directed subsets and \(\omega\)-chains.
\item \cref{exer:cantor-domain}: On the \(\omega\)-cppo of finite and infinite
  binary sequences.
\item \cref{exer:monotonicity-follows}: On deriving monotonicity from preservation of
  least upper bounds of \(\omega\)-chains.
\item \cref{exer:least-fixed-point}: On showing that the Knaster--Tarski fixed point
  is the least.
\item \cref{exer:category-of-cpos}: On the category of \(\omega\)-cpos.
\item \cref{exer:adjunctions}: On adjunctions between the categories of sets and
  (pointed) \(\omega\)-cpos.
\item \cref{exer:continuity-in-each-argument}: On checking \(\omega\)-continuity
  in each argument of the product.
\item \cref{exer:product-induced}: On \(\omega\)-continuity of the induced map
  to the product.
\item \cref{exer:approximating-successor}: On approximating the successor map on
  \(\Nat_\bot\).
\item \cref{exer:evaluation-is-continuous}: On \(\omega\)-continuity of the evaluation map.
\item \cref{exer:curry-is-continuous}: On \(\omega\)-continuity of the curried map.
\end{enumerate}

%%% Local Variables:
%%% mode: latexmk
%%% TeX-master: "../main"
%%% End:
