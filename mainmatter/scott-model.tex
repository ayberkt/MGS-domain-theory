\chapter{The Scott model of PCF}

We started these notes by introducing PCF and its operational semantics
in~\cref{chap:PCF}. We then developed sufficient domain theory
in~\cref{chap:domains} to now give a denotational semantics of PCF, known as the
Scott model of PCF, where we interpret the types of PCF as \(\omega\)-cppos and
terms as \(\omega\)-continuous maps.

\begin{definition}[Interpretation of PCF types, \(\densem{\sigma}\)]
  We inductively assign an \(\omega\)-cppo \(\densem{\sigma}\) to each type
  \(\sigma\) of PCF:
  \[
    \densem{\pcfnat} \coloneqq \Nat_\bot
    \quad\text{and}\quad
    \densem{\sigma \pcffun \tau} \coloneqq \densem{\tau}^{\densem{\sigma}},
  \]
  % \begin{alignat*}{3}
  %   &\densem{\pcfnat} &&\coloneqq \Nat_\bot, \text{and} \\
  %   &\densem{\sigma \pcffun \tau} &&\coloneqq \densem{\tau}^{\densem{\sigma}},
  % \end{alignat*}
  where we recall \(\Nat_\bot\) from~\cref{exam:N_bot} and the exponential
  from~\cref{def:exponential}.
\end{definition}

This interpretation extends to contexts by using iterated binary products:
\begin{definition}[Interpretation of contexts, \(\densem{\Gamma}\)]
  The interpretation of a context
  \(\Gamma = [\var x_0 : \sigma_0, \dots , \var x_{n-1} :
  \sigma_{n-1}]\)
  is
  \[
    \densem{\Gamma} \coloneqq \densem{\sigma_0} \times \cdots \times
    \densem{\sigma_{n-1}}
  \]
  with the convention that the empty context is interpreted as the empty
  product: the one point \(\omega\)-cppo \(\One\) from~\cref{exam:one-point}.
\end{definition}

The interpretation of the terms of PCF proceeds by yet another inductive
definition:

\begin{definition}[Interpretation of PCF terms, \(\densem{M}\)]\label{def:interpretation}
  A term \(\Gamma \vdash M : \sigma\) of type \(\sigma\) in context~\(\Gamma\)
  will be interpreted as an \emph{\(\omega\)-continuous} function
  \[
    \densem{\Gamma} \xrightarrow{\densem{M}} \densem{\sigma}
  \]
  according to the following clauses which mirror~\cref{def:PCF-terms}:
  \begin{enumerate}[(i)]
  \item
    The interpretation of
    \[
      \def\fCenter{\ \vdash\ }
      \AxiomC{\phantom{$\fCenter$}}
      \UnaryInf$\Gamma,\var{x}:\sigma,\Delta \fCenter \var{x} : \sigma$
      \DisplayProof
    \]
    is
    \[
      \densem{\Gamma} \times \densem{\sigma} \times {\densem{\Delta}}
      \xrightarrow{\pi} \densem{\sigma},
    \]
    where \(\pi\) is a suitable composition of projections
    (recall~\cref{projections}), which is \(\omega\)-continuous, because
    \(\omega\)-continuous functions are closed under composition
    (recall~\cref{exer:category-of-cpos}).
  \item
    To interpret
    \[
      \def\fCenter{\ \vdash\ }
      \Axiom$\Gamma , \var{x} : \sigma \fCenter M : \tau$
      \UnaryInf$\Gamma \fCenter (\lambdadot{\var{x} : \sigma}{M}) : \sigma
      \pcffun \tau$ \DisplayProof
    \]
    we first remark that we have
    \[
      \densem{\Gamma} \times \densem{\sigma} \xrightarrow{\densem{M}} \densem{\tau}
    \]
    by induction hypothesis, so that we can define
    \[
      \densem{\Gamma} \xrightarrow{\densem{\lambdadot{\var x : \sigma}{M}}}
      \densem{\tau}^{\densem{\sigma}}
    \]
    as the curried (recall~\cref{exer:curry-is-continuous}) version of
    \(\densem{M}\), i.e.\ for \(\gamma \in \densem{\Gamma}\), we have
    \(\densem{\lambdadot{\var x : \sigma}{M}}(\gamma) \coloneqq x \mapsto
    \densem{M}(\gamma,x)\).
  \item
    To interpret
    \[
      \def\fCenter{\ \vdash\ }
      \Axiom$\Gamma \fCenter M : \sigma \pcffun \tau$
      \Axiom$\Gamma \fCenter N : \sigma$
      \BinaryInf$\Gamma \fCenter M \, N : \tau$
      \DisplayProof
    \]
    we first remark that we have
    \[
      \densem{\Gamma} \xrightarrow{\densem{M}} \densem{\tau}^{\densem{\sigma}}
      \quad\text{and}\quad
      \densem{\Gamma} \xrightarrow{\densem{N}} \densem{\sigma}
    \]
    by induction hypothesis, so that we can define
    \[
      \densem{\Gamma} \xrightarrow{\densem{M \, N}} \densem{\tau}
    \]
    by application: for \(\gamma \in \densem{\Gamma}\), we have
    \(\densem{M \, N}(\gamma) \coloneqq
    \densem{M}(\gamma)\pa*{\densem{N}(\gamma)}\).

    More abstractly, it is the composition of
    \[
      \densem{\Gamma} \xrightarrow{\langle \densem{M} , \densem{N} \rangle}
      \densem{\tau}^{\densem{\sigma}} \times \densem{\sigma}
      \xrightarrow{\text{evaluation}}
      \densem{\tau},
    \]
    where we recall~\cref{exer:product-induced,evaluation-is-continuous}.
  \item
    To interpret
    \[
      \def\fCenter{\ \vdash\ }
      \Axiom$\Gamma \fCenter M : \sigma \pcffun \sigma$
      \UnaryInf$\Gamma \fCenter \pcffix_\sigma(M) : \sigma$ \DisplayProof
    \]
    we first remark that we have
    \[
      \densem{\Gamma} \xrightarrow{\densem{M}} \densem{\sigma}^{\densem{\sigma}}
    \]
    by induction hypothesis, so that we can define
    \[
      \densem{\Gamma} \xrightarrow{\densem{\pcffix_\sigma\,M}} \densem{\sigma}
    \]
    as the composition
    \(\densem{\Gamma} \xrightarrow{\densem{M}} \densem{\sigma}^{\densem{\sigma}}
    \xrightarrow{\mu} \densem{\sigma}\), where we recall \(\mu\), which assigns
    least fixed points, from~\cref{least-fixed-point-is-continuous}.
  \item
    The interpretation of
    \[
      \def\fCenter{\ \vdash\ }
      \AxiomC{}
      \UnaryInf$\Gamma \fCenter \pcfzero : \pcfnat$
      \DisplayProof
    \]
    is
    \[
      \densem{\Gamma} \xrightarrow{\densem{\pcfzero}} \Nat_\bot
    \]
    which is given by \(\gamma \in \densem{\Gamma} \mapsto 0 \in \Nat_\bot\).
  \item\label{def:interpretation-succ}
    To interpret
    \[
      \def\fCenter{\ \vdash\ }
      \Axiom$\Gamma \fCenter M : \pcfnat$
      \UnaryInf$\Gamma \fCenter \pcfsuc(M) : \pcfnat$
      \DisplayProof
    \]
    we first remark that we have
    \[
      \densem{\Gamma} \xrightarrow{\densem{M}} \Nat_\bot
    \]
    by induction hypothesis, so that we can define
    \[
      \densem{\Gamma} \xrightarrow{\densem{\pcfsuc\,M}} \Nat_\bot
    \]
    as the composition
    \(\densem{\Gamma} \xrightarrow{\densem{M}} \Nat_\bot \xrightarrow{s}
    \Nat_\bot\), where \(s : \Nat_\bot \to \Nat_\bot\) is the successor function
    on \(\Nat_\bot\), i.e.\ \(s(\bot) \coloneqq \bot\) and
    \(s(n) \coloneqq n+1\).
  \item\label{def:interpretation-pred}
    To interpret
    \[
      \def\fCenter{\ \vdash\ }
      \Axiom$\Gamma \fCenter M : \pcfnat$
      \UnaryInf$\Gamma \fCenter \pcfpred(M) : \pcfnat$
      \DisplayProof
    \]
    we first remark that we have
    \[
      \densem{\Gamma} \xrightarrow{\densem{M}} \Nat_\bot
    \]
    by induction hypothesis, so that we can define
    \[
      \densem{\Gamma} \xrightarrow{\densem{\pcfpred\,M}} \Nat_\bot
    \]
    as the composition
    \(\densem{\Gamma} \xrightarrow{\densem{M}} \Nat_\bot \xrightarrow{p}
    \Nat_\bot\), where \(p : \Nat_\bot \to \Nat_\bot\) is the predecessor
    function on \(\Nat_\bot\), i.e.\ \(p(\bot) \coloneqq \bot\),
    \(p(0) \coloneqq 0\) and \(p(n+1) \coloneqq n\).
  \item\label{def:interpretation-ifzero}
    Finally, to interpret
    \[
      \def\fCenter{\ \vdash\ }
      \Axiom$\Gamma \fCenter M : \pcfnat$
      \Axiom$\Gamma \fCenter N_1 : \pcfnat$
      \Axiom$\Gamma \fCenter N_2 : \pcfnat$
      \TrinaryInf$\Gamma \fCenter \pcfifz(M,N_1,N_2) : \pcfnat$
      \DisplayProof
    \]
    we first remark that we have
    \[
      \densem{\Gamma} \xrightarrow{\densem{M}} \Nat_\bot,
      \densem{\Gamma} \xrightarrow{\densem{N_1}} \Nat_\bot \text{ and }
      \densem{\Gamma} \xrightarrow{\densem{N_2}} \Nat_\bot
    \]
    by induction hypothesis, so that we can define
    \[
      \densem{\Gamma} \xrightarrow{\densem{\pcfifz(M,N_1,N_2)}} \Nat_\bot
    \]
    as the composition
    \(\densem{\Gamma}
    \xrightarrow{\langle{\densem{M},\densem{N_1},\densem{N_2}}\rangle}
    {{\Nat_\bot} \times {\Nat_\bot} \times {\Nat_\bot}} \xrightarrow{c}
    \Nat_\bot\), where
    \[
      c : \Nat_\bot \times \Nat_\bot \times \Nat_\bot \to \Nat_\bot
    \] is defined as the \(\omega\)-continuous function
    \(c(\bot,y,z) \coloneqq \bot\), \(c(0,y,z) \coloneqq y\) and
    \(c(n+1,y,z) \coloneqq z\). \qedhere
  \end{enumerate}
\end{definition}

\begin{remark}[Programs of type \(\sigma\) are interpreted as elements of
  \(\densem{\sigma}\)]
  Note that a program, i.e.\ a closed term, \({} \vdash M : \sigma\) is
  interpreted as a function
  \(\densem{M} \colon \One \to \densem{\sigma}\).
  %
  Since the underlying set of \(\One\) is the singleton \(\set{\star}\), this
  function simply picks out an element. % of \(\densem{\sigma}\).
  %
  With this in mind, we see that programs of type \(\sigma\) are interpreted
  as elements of \(\densem{\sigma}\).
\end{remark}


\begin{exercise}\label{exer:interpretation-of-numerals}
  Show that the numerals of PCF are interpreted as the natural numbers in
  \(\Nat_\bot\), i.e.\ \(\densem{\numeral{n}} = n \in \Nat_\bot\) for every
  natural number \(n \in \Nat\).
\end{exercise}

\section{Towards soundness}

The next section will be devoted to proving the \emph{soundness} theorem, which
says that if \(M \bigstep N\), then \(\densem{M} = \densem{N}\). In other words,
the interpretation respects the operational semantics.
%
The proof of soundness relies on the following lemma:% , which is actually a
% particular case of the fact that \(M \smallstep N\) implies
% \(\densem{M} = \densem{N}\).

\begin{lemma}[\(\beta\)-equality]\label{beta-equality}
  The Scott model of PCF validates \(\beta\)-equality. That is, if
  \(\Gamma,\var x : \sigma \vdash M : \tau\) and \(\Gamma \vdash N : \sigma\),
  then
  \[
    \densem*{\pa*{\lambdadot{\var x : \sigma}{M}}\,N} = \densem{M[N/\var x]}
  \]
\end{lemma}

\begin{exercise}\label{exer:beta-equality}
  Prove~\cref{beta-equality} from the substitution lemma given below.
\end{exercise}


Suppose that we have a term \(M : \tau\) in context
\(\Gamma = [\var x_0 : \sigma_0 , \dots , \var x_{k-1} : \sigma_{k-1}]\), and
assume that we have another context \(\Delta\) with terms
\(\Delta \vdash N_i : \sigma_i\) for every \(0 \leq i \leq k-1\).
%
This yields (by a recursive definition on the structure of derivations of
\(\Gamma \vdash M : \tau\)) a term
\[
  \Delta \vdash M[N_0/\var x_0,\dots,N_{k-1}/\var x_{k-1}] : \tau
\]
in context \(\Delta\).

Thus, in our interpretation, we get
\begin{equation*}\label{subst-1}\tag{\(\dagger\)}
  \densem{M[N_0/\var x_0,\dots,N_{k-1}/\var x_{k-1}]}(\delta) \in \densem{\tau}
\end{equation*}
for every \(\delta \in \densem{\Delta}\).

Alternatively, we can note that \(\densem{N_i}(\delta) \in \densem{\sigma_i}\)
for every \(0 \leq i \leq k-1\), so that we can feed them as inputs to
\(\densem{M} \colon \densem{\sigma_0} \times \cdots \times \densem{\sigma_{k-1}}
\to \densem{\tau}\) which yields:
\begin{equation*}\label{subst-2}\tag{\(\ddagger\)}
  \densem{M}\pa*{\densem{N_0}(\delta),\dots,\densem{N_{k-1}}(\delta)} \in \densem{\tau}.
\end{equation*}

The substitution lemma says that \eqref{subst-1} and \eqref{subst-2} agree.

\begin{lemma}[Substitution lemma]\label{substitution-lemma}
  If
  \([\var x_0 : \sigma_0,\dots,\var x_{k-1} : \sigma_{k-1}] = \Gamma \vdash M :
  \tau\), then for every context \(\Delta\) and terms
  \(\Delta \vdash N_i : \sigma_i\) with \(0 \leq i \leq k-1\), we have
  \[
    \densem{M[N_0/\var x_0,\dots,N_{k-1}/\var x_{k-1}]}(\delta)
    =
    \densem{M}\pa*{\densem{N_0}(\delta),\dots,\densem{N_{k-1}}(\delta)}
  \]
  for every \(\delta \in \densem{\Delta}\).
\end{lemma}
\begin{proof}
  By induction on the structure of derivations of \(\Gamma \vdash M : \tau\).

  For example, for the case of \(\lambda\)-abstraction, we consider the term
  \(\Gamma,y : \tau \vdash M : \rho\) with
  \([\var x_0 : \sigma_0,\dots,\var x_{k-1} : \sigma_{k-1}] = \Gamma\), and we
  assume to have a context \(\Delta\) with terms
  \(\Delta \vdash N_i : \sigma_i\) for \(0 \leq i \leq k-1\).
  We have to prove that
  \[
    \densem*{\pa*{\lambdadot{\var y : \tau}{M}}[N_0/\var x_0 , \dots ,
      N_{k-1}/\var x_{k-1}]}(\delta) = \densem*{\lambdadot{\var y :
        \tau}{M}}\pa*{\densem{N_0}(\delta),\dots,\densem{N_{k-1}}(\delta)}
  \]
  for every \(\delta \in \densem{\Delta}\).

  Note that this is an equality of elements of
  \(\densem{\rho}^{\densem{\tau}}\), i.e.\ an equality of
  (\(\omega\)-continuous) functions, so let \(t \in \densem{\tau}\) be
  arbitrary and note that
  \begin{align*}
    &\hspace{13.5pt} \pa*{\densem*{\pa*{\lambdadot{\var y : \tau}{M}}[N_0/\var x_0 , \dots ,
    N_{k-1}/\var x_{k-1}]}(\delta)}(t) \\
    &= \densem*{M[N_0/\var x_0,\dots,N_{k-1}/\var x_{k-1},\var y/\var y]}(\delta,t) \\
    &= \densem*{M}\pa*{\densem{N_0}(\delta,t),\dots,\densem{N_{k-1}}(\delta,t),\densem{\var y}(\delta,t)}
    &\text{(by IH applied to \(\Gamma,\var y : \tau \vdash M : \rho\))} \\
    &= \densem*{M}\pa*{\densem{N_0}(\delta,t),\dots,\densem{N_{k-1}}(\delta,t),t} \\
    &= \pa*{\densem*{\lambdadot{\var y :
        \tau}{M}}\pa*{\densem{N_0}(\delta),\dots,\densem{N_{k-1}}(\delta)}}(t),
  \end{align*}
  as desired.
  %
  Note that the \(N_i\) in lines 2--4 refer to the terms \(N_i\) in the extended
  context \(\Delta,\var y : \tau\).
\end{proof}

\section{Soundness}\label{sec:soundness}

The soundness theorem expresses that the denotational semantics, in the form of
the Scott model of PCF, respects the operational semantics.

\begin{theorem}[Soundness]\label{soundness}
  For terms \(M\) and \(N\) of the same type in the same context, we have: if
  \(M \bigstep N\), then \(\densem{M} = \densem{N}\).
\end{theorem}
\begin{proof}
  By induction on the structure of the derivations of \(M \bigstep N\).
  %
  We work out two cases: the fixed point operator and application.

  \begin{itemize}
  \item Recall that the former is the rule
  \[
    \AxiomC{\(M(\pcffix_{\sigma} \, M) \bigstep V\)}
    \UnaryInfC{\({\pcffix_{\sigma} \, M} \bigstep V\)}
    \DisplayProof
  \]
  so that we may assume \(\densem{M(\pcffix_\sigma\,M)} = \densem{V}\) and we
  have to prove:
  \[
    \densem{\pcffix_{\sigma} \, M} = \densem{V}.
  \]
  But this holds: if \(M\) is a term in context \(\Gamma\), then for
  every \(\gamma \in \densem{\Gamma}\), we have
  \begin{align*}
    \densem{\pcffix_{\sigma} \,M}(\gamma)
    &= \mu\pa*{\densem{M}(\gamma)}
    &\text{(by definition)} \\
    &= \pa*{\densem{M}(\gamma)}\pa*{\mu\pa*{\densem{M}(\gamma)}}
    &\text{(because \(\mu\) gives a fixed point)} \\
    &= \densem{M\pa*{\pcffix_{\sigma}\,M}}(\gamma)
    &\text{(by definition)} \\
    &= \densem{V}(\gamma) &\text{(by assumption)}.
  \end{align*}
  \item   The big-step rule for application is
  \[
    \AxiomC{\(M \bigstep \lambdadot{\var x : \sigma}{E}\)}
    \AxiomC{\(E[N/x] \bigstep V\)}
    \BinaryInfC{\(M \, N \bigstep V\)}
    \DisplayProof
  \]
  so that we may assume
  \(\densem{M} = \densem{\lambdadot{\var x : \sigma}{E}}\)
  and
  \(\densem{E[N/x]} = \densem{V}\),
  and we have to prove:
  \[
    \densem{M \, N} = \densem{V}.
  \]
  If \(M\) and \(N\) are terms in a context \(\Gamma\), then for every
  \(\gamma \in \densem{\Gamma}\), we calculate:
  \begin{align*}
    \densem{M \, N}(\gamma)
    &= \densem{M}(\gamma)\pa*{\densem{N}(\gamma)}
    &\text{(by definition)} \\
    &= \densem{\lambdadot{\var x : \sigma}{E}}(\gamma)\pa*{\densem{N}(\gamma)}
    &\text{(by assumption on \(\densem{M}\))} \\
    &= \densem*{\pa*{\lambdadot{\var x : \sigma}{E}}N}(\gamma)
    &\text{(by definition)} \\
    &= \densem{E[N/\var x]}(\gamma)
    &\text{(by \cref{beta-equality})} \\
    &= \densem{V}(\gamma)
    &\text{(by assumption on \(\densem{N}\))},
  \end{align*}
  completing the proof. \qedhere
  \end{itemize}
\end{proof}

\begin{exercise}\label{exer:non-termination-denotational}
  % Recall the program \(M \coloneqq \pcffix_{\pcfnat}(S) : \pcfnat\) with
  % \(S \coloneqq \pa*{\lambdadot{\var x : \pcfnat}{\pcfsuc \, \var x}} : \pcfnat
  % \pcffun \pcfnat\) from~\cref{exam:non-termination}.

  Consider the program
  \(M \coloneqq \pcffix_\pcfnat (\lambdadot{\var x : \pcfnat}{\var x})\).

  Prove that \(\densem{M} = \bot\) and use the soundness theorem to conclude
  that \(M\) does not reduce to a value.

  \emph{Hint}: For a particularly quick proof that \(\densem{M} = \bot\), use
  \cref{exer:least-fixed-point}.
\end{exercise}

\section{List of exercises}
\begin{enumerate}
\item \cref{exer:interpretation-of-numerals}: On the interpretation of the numerals.
\item \cref{exer:beta-equality}: On proving that the interpretation validates
  \(\beta\)-equality.
\item \cref{exer:non-termination-denotational}: On using the soundness theorem
  to prove non-termination.
\end{enumerate}

%%% Local Variables:
%%% mode: latexmk
%%% TeX-master: "../main"
%%% End:
