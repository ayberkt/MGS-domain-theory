\chapter{PCF and its operational semantics}\label{chap:PCF}

PCF (Programming Computable Functions)~\cite{Plotkin1977} is a typed functional
programming language with general recursion. We can think of PCF as a simpler,
well-behaved fragment of a modern functional programming language such as
Haskell~\cite{Haskell2010}.
%
The point of PCF is not to be a particularly convenient or rich programming
language. Instead, it's meant to be a simple and principled language enabling us
to study it from a mathematical viewpoint without having to deal with complex
features that you might find in modern, real-world programming languages.
%
The techniques that we will employ could, with some effort, be extended to more
complex programming languages however, see e.g.~\cite{Plotkin1983}.

At the same time, it should be mentioned that PCF is not quite a toy language
and has a rather rich theory. For example, it captures Kleene--Kreisel
higher-type computability~\cite{LongleyNormann2015} and an extension of
PCF---with parallel-or and \(\exists\) (see~\cite{Streicher2006} for
details)---can simulate recursively defined datatypes (such as trees and
lists)~\cite{Streicher1994}.

\section{PCF}

We describe the syntax of PCF and its small-step operational semantics which
describe how to compute in PCF.

\begin{definition}[Types of PCF, \(\pcfnat\), \(\sigma \pcffun \tau\)]
  The \emph{types of PCF} are inductively defined as:
  \begin{enumerate}[(i)]
  \item \(\pcfnat\) is the \emph{base type} of PCF, and
  \item if \(\sigma\) and \(\tau\) are types of PCF, then we have the
    \emph{function type} \(\sigma \pcffun \tau\).
  \end{enumerate}
  Moreover, as usual, we will write \(\sigma \pcffun \tau \pcffun \rho\) for
  \(\sigma \pcffun (\tau \pcffun \rho)\).
\end{definition}

From now on, such inductive definitions will be presented in the following style:
\begin{center}
  \AxiomC{\phantom{\(\pcfnat\)}}
  \UnaryInfC{\(\pcfnat\) is a type of PCF}
  \DisplayProof\hspace{3cm}
  \AxiomC{\(\sigma\) is a type of PCF}
  \AxiomC{\(\tau\) is a type of PCF}
  \BinaryInfC{\(\pa*{\sigma \pcffun \tau}\) is a type of PCF}
  \DisplayProof
\end{center}

For each type \(\sigma\), we assume to have a countably infinite set of typed
variables, typically denoted by \(\var x : \sigma\), \(\var y : \sigma\), or
\(\var x_1 : \sigma\), \(\var x_2 : \sigma\), etc.

\begin{definition}[Context, \(\Gamma\)]\label{def:context}
  A \emph{context} \(\Gamma\) is a list of variables:
  \[
    \Gamma = [\var x_0 : \sigma_0 , \var x_1 : \sigma_1 , \dots , \var x_{n-1} :
    \sigma_{n-1}].
  \]
  For \(n = 0\), we get the \emph{empty context}: an empty list with no
  variables.
\end{definition}

We are now ready to define the (well-typed) terms of PCF.

\begin{definition}[Terms of PCF, \(\Gamma \vdash M : \sigma\)]\label{def:PCF-terms}
  We inductively define the \emph{terms \(M\) of type~\(\sigma\) in
    context~\(\Gamma\)}, written \(\Gamma \vdash M : \sigma\), by the following
  clauses:
  \begin{center}
  \def\fCenter{\ \vdash\ }

  \AxiomC{\phantom{$\fCenter$}}
  \UnaryInf$\Gamma,\var{x}:\sigma,\Delta \fCenter \var{x} : \sigma$
  \DisplayProof\hspace{3.7cm}
  \Axiom$\Gamma , \var{x} : \sigma \fCenter M : \tau$
  \UnaryInf$\Gamma \fCenter (\lambdadot{\var{x} : \sigma}{M}) : \sigma \pcffun \tau$
  \DisplayProof\vspace{1cm}\\
  \Axiom$\Gamma \fCenter M : \sigma \pcffun \tau$
  \Axiom$\Gamma \fCenter N : \sigma$
  \BinaryInf$\Gamma \fCenter M(N) : \tau$
  \DisplayProof\hspace{3.9cm}
  \Axiom$\Gamma \fCenter M : \sigma \pcffun \sigma$
  \UnaryInf$\Gamma \fCenter \pcffix_\sigma(M) : \sigma$
  \DisplayProof\vspace{1cm}\\
  \AxiomC{\vphantom{$\Gamma$}} %% putting a phantom \Gamma here ensures vertical alignment
  \UnaryInf$\Gamma \fCenter \pcfzero : \pcfnat$
  \DisplayProof\quad\quad\quad
  \Axiom$\Gamma \fCenter M : \pcfnat$
  \UnaryInf$\Gamma \fCenter \pcfsuc(M) : \pcfnat$
  \DisplayProof\quad\quad\quad
  \Axiom$\Gamma \fCenter M : \pcfnat$
  \UnaryInf$\Gamma \fCenter \pcfpred(M) : \pcfnat$
  \DisplayProof\vspace{1cm}\\
  \Axiom$\Gamma \fCenter M : \pcfnat$
  \Axiom$\Gamma \fCenter N_1 : \pcfnat$
  \Axiom$\Gamma \fCenter N_2 : \pcfnat$
  \TrinaryInf$\Gamma \fCenter \pcfifz(M,N_1,N_2) : \pcfnat$
  \DisplayProof
\end{center}
When \(\Gamma\) is the empty context, we simply write \(\vdash M : \sigma\) and
we call \(M\) a \emph{closed} term or a \emph{program}.
%
Note that programs do not contain any free variables.

We will often write \(M \, N\) instead of \(M(N)\) to ease readability.
\end{definition}

The first three rules above will look familiar to someone who has seen the typed
\(\lambda\)-calculus before and basically say that we form functions that we can
apply. The three rules on the third row give us natural numbers with a
predecessor constructor.
%
We illustrate the remaining rules, those for \(\pcffix\) and \(\pcfifz\),
in~\cref{exam:ifzero,exam:addition-by-n} below.

So far, we only have some terms, but we have not specified any computational
behaviour of those terms yet. \cref{def:small-step} defines a reduction strategy
that specifies the computational behaviour of PCF and should help us understand
the intended meaning of the terms.

\begin{definition}[Numeral, \(\numeral{n}\)]
  For a natural number \(n \in \Nat\), we define the
  \emph{numeral}~\(\numeral n\) in PCF inductively:
  \(\numeral 0 \coloneqq \pcfzero\) and
  \(\numeral{m+1} \coloneqq \pcfsuc \, \numeral m\).
\end{definition}

\begin{definition}[Small-step operational semantics of PCF, \(M \smallstep N\)]%
  \label{def:small-step}%
  We inductively define when a term \(M\) \emph{(small-step) reduces} to another
  term \(N\) (of the same type, in the same context), written
  \(M \smallstep N\), by the following inductive clauses:
  %\begin{center}
    \begin{longtable}{cc}
      \AxiomC{\phantom{${\smallstep}$}}
      \UnaryInfC{\((\lambdadot{\var x : \sigma}{M})N \smallstep M[N/\var x]\)}
      \DisplayProof
      &
        \AxiomC{\phantom{${\smallstep}$}}
        \UnaryInfC{\(\pcffix_\sigma \, M \smallstep M\pa*{\pcffix_\sigma \, M}\)}
        \DisplayProof\vspace{.5cm}\\
      \AxiomC{\phantom{${\smallstep}$}}
      \UnaryInfC{\(\pcfpred \, \numeral 0 \smallstep \numeral 0\)}
      \DisplayProof
      &
      \AxiomC{\phantom{${\smallstep}$}}
      \UnaryInfC{\(\pcfpred \, \numeral{n+1} \smallstep \numeral n\)}
      \DisplayProof\vspace{.5cm}\\
      \AxiomC{\phantom{${\smallstep}$}}
      \UnaryInfC{\(\pcfifz(\numeral 0,M,N) \smallstep M\)}
      \DisplayProof
      &
      \AxiomC{\phantom{${\smallstep}$}}
      \UnaryInfC{\(\pcfifz(\numeral {n+1},M,N) \smallstep N\)}
      \DisplayProof\vspace{.5cm}\\
      \AxiomC{\(M_1 \smallstep M_2\)}
      \UnaryInfC{\(M_1 \, N \smallstep M_2 \, N\)}
      \DisplayProof
      &
      \AxiomC{\(M_1 \smallstep M_2\)}
      \UnaryInfC{\(\pcfsuc \, M_1 \smallstep \pcfsuc \, M_2\)}
      \DisplayProof\vspace{.5cm}\\
      \AxiomC{\(M_1 \smallstep M_2\)}
      \UnaryInfC{\(\pcfpred \, M_1  \smallstep \pcfpred \, M_2\)}
      \DisplayProof
      &
      \AxiomC{\(M_1 \smallstep M_2\)}
      \UnaryInfC{\(\pcfifz(M_1,N_1,N_2) \smallstep \pcfifz(M_2,N_1,N_2)\)}
      \DisplayProof
    \end{longtable}
  %\end{center}
  Here \(M[N/x]\) denotes the result of substituting \(N\) for the variable
  \(\var x\) in \(M\).
\end{definition}

The final three rules are like congruence rules saying that you can reduce a
term \(\pcfsuc M\) by reducing the inner term \(M\).
%
The reduction \(\pcffix_{\sigma}\,M \smallstep M(\pcffix_\sigma \, M)\)
corresponds to the idea that \(\pcffix_\sigma\,M\) is a \emph{fixed point} of
\(M : \sigma \pcffun \sigma\). Alternatively, we can think of it as a single
unfolding of a recursive definition, as illustrated
in~\cref{exer:small-step-addition,exam:non-termination}.

\begin{example}\label{exam:ifzero}
  We can translate the program \verb|if (x + 1 == 0) then 5 else 3|, written in
  pseudocode, to the PCF program
  \(\lambdadot{\var x : \pcfnat}{\pcfifz(\pcfsuc \, \var x, \numeral 5, \numeral 3)}\).
\end{example}



The point of \(\pcffix\) is that it gives us general recursion, as we will
explain with an example now.

\begin{example}[Addition by \(n\) in PCF]\label{exam:addition-by-n}
  For a natural number \(n \in \Nat\), consider addition by \(n\) as a
  recursively defined function:
  \begin{alignat*}{3}
    &\add_n : \Nat \to && \Nat && \\
    &\add_n(0) &&\coloneqq n, && \\
    &\add_n(k+1) &&\coloneqq \add_n(k) + 1. &&
  \end{alignat*}
  We show how to write this function as a PCF program. We start by slightly
  rewriting \(\add_n\) as:
  \begin{equation*}\label{add_n-alt}\tag{\(\ast\)}
    \add_n(m) \coloneqq
    \begin{cases}
      n &\text{if } m = 0, \\
      \suc\pa*{\add_n(\pred(m))} &\text{else}.
    \end{cases}
  \end{equation*}
  Looking at the above, we see that we have most of the analogues readily
  available in PCF: \(\pcfifz\), \(\pcfsuc\) and \(\pcfpred\), as well as the
  numeral \(\underline n\).

  We next define the program
  \(F : \pa*{\pcfnat \pcffun \pcfnat} \pcffun \pa*{\pcfnat \pcffun \pcfnat}\) as
  follows:
  \[
    F \coloneqq \lambdadot{\var f : \pa{\pcfnat \pcffun
        \pcfnat}}{\lambdadot{\var y : \pcfnat}%
      {\pcfifz\pa*{\var y,\numeral{n},%
          \pcfsuc\pa*{\var f\pa*{\pcfpred \, \var y }}}}}.
  \]
  In the program \(F\), the variable \(\var y\) plays the role of \(m\) in
  \eqref{add_n-alt} and \(\var f\) is like a placeholder for the recursive call.

  Mirroring~\eqref{add_n-alt} further, what we want is a program
  \(f : \pcfnat \pcffun \pcfnat\) such that \(f\) is ``equal'' to \(F\,f\). That
  is, we want a \emph{fixed point} of \(F\).
  %
  Hence, we finally define \(\pcfadd_n\) as:
  \[
    \pcfadd_n \coloneqq \pcffix_{\pcfnat \pcffun \pcfnat} \, F. \qedhere
  \]
  % \[
  %   \pcffix_{\pcfnat \pcffun \pcfnat} : \pa*{\pa*{\pcfnat \pcffun \pcfnat}
  %     \pcffun \pa*{\pcfnat \pcffun \pcfnat}} \pcffun \pa{\pcfnat \pcffun \pcfnat}.
  % \]
\end{example}

\begin{exercise}\label{exer:small-step-addition}
  Give sequences of small-step reductions showing that
  \(\pcfadd_n \, \numeral 0\) and \(\pcfadd_n \, \numeral 1\) respectively
  compute to the numerals \(\numeral n\) and \(\numeral {n+1}\).
\end{exercise}

\begin{exercise}\label{exer:pcf-add-mult}
  Construct PCF programs
  \(\pcfadd,\pcffont{mult} : \pcfnat \pcffun \pcfnat \pcffun \pcfnat\)
  implementing addition and multiplication, respectively.
\end{exercise}

Having general recursion also means that we have non-terminating programs, such
as the following one:

\begin{example}\label{exam:non-termination}
  Define
  \(S \coloneqq \pa*{\lambdadot{\var x : \pcfnat}{\pcfsuc \, \var x}} : \pcfnat
  \pcffun \pcfnat\) and consider the PCF program
  \(M \coloneqq \pcffix_{\pcfnat}(S) : \pcfnat\).  Repeatedly applying the
  small-step operational semantics, we get:
  \begin{alignat*}{3}
    M &= \pcffix_{\pcfnat} \, S \\
    &\smallstep
       S(\pcffix_{\pcfnat} \, S)\\
    &\smallstep
      \pcfsuc(\pcffix_{\pcfnat} \, S)
    &&= \pcfsuc \, M \\
    &\smallstep
      \pcfsuc\pa*{S\pa*{\pcffix_{\pcfnat} \, S}} \\
    &\smallstep
      \pcfsuc\pa*{\pcfsuc\pa*{\pcffix_{\pcfnat} \, S}}
    &&= \pcfsuc\pa*{\pcfsuc\, M} \\
    &\smallstep \dots
  \end{alignat*}
  Thus, the term \(M : \pcfnat\) never computes to a numeral.
\end{example}

\section{Big-step operational semantics}

In our investigations into the denotational semantics of PCF, we will typically
not be interested in intermediate reduction steps, e.g.\ we don't really care
about the whole sequence
\({\pcfifz(\pcfpred\,\numeral{3},\numeral{5},\pcfpred\,\numeral{7})} \smallstep
{\pcfifz(\numeral{2},\numeral{5},\pcfpred\,\numeral{7})} \smallstep
{\pcfpred\,\numeral{7}} \smallstep {\numeral{6}}\); we only care that
\(\pcfifz(\pcfpred\,\numeral{3},\numeral{5},\pcfpred\,\numeral{7})\) computes to
the numeral \(\numeral{6}\).
%
The big-step operational semantics of PCF is a way of formalising this
desideratum.

\begin{definition}[Big-step operational semantics of PCF, \(M \bigstep V\)]
  We inductively define when a term \(M\) \emph{(big-step) reduces} to another
  term \(V\) (of the same type, in the same context), written
  \(M \bigstep V\), by the following inductive clauses:
  \begin{longtable}{cc}
  \AxiomC{\phantom{${\bigstep}$}}
  \UnaryInfC{\(\var x \bigstep \var x\)}
  \DisplayProof
  &
  \AxiomC{\phantom{${\bigstep}$}}
  \UnaryInfC{\(\lambdadot{\var x : \sigma}{M} \bigstep \lambdadot{\var x : \sigma}{M}\)}
  \DisplayProof\vspace{1cm}\\
  \AxiomC{\(M \bigstep \lambdadot{\var x : \sigma}{E}\)}
  \AxiomC{\(E[N/x] \bigstep V\)}
  \BinaryInfC{\(M \, N \bigstep V\)}
  \DisplayProof
  &
  \AxiomC{\(M(\pcffix_{\sigma} \, M) \bigstep V\)}
  \UnaryInfC{\({\pcffix_{\sigma} \, M} \bigstep V\)}
  \DisplayProof\vspace{1cm}\\
  \AxiomC{\phantom{${\bigstep}$}}
  \UnaryInfC{\(\numeral 0 \bigstep \numeral 0\)}
  \DisplayProof
  &
  \AxiomC{\(M \bigstep \numeral{n}\)}
  \UnaryInfC{\(\pcfsuc \, M \bigstep \numeral{n+1}\)}
  \DisplayProof\vspace{1cm}\\
  \AxiomC{\(M \bigstep \numeral 0\)}
  \UnaryInfC{\(\pcfpred \, M \bigstep \numeral 0\)}
  \DisplayProof
  &
  \AxiomC{\(M \bigstep \numeral {n+1}\)}
  \UnaryInfC{\(\pcfpred \, M \bigstep \numeral n\)}
  \DisplayProof\vspace{1cm}\\
  \AxiomC{\(M \bigstep \numeral 0\)}
  \AxiomC{\(N_1 \bigstep V\)}
  \BinaryInfC{\(\pcfifz(M,N_1,N_2) \bigstep V\)}
  \DisplayProof
  &
  \AxiomC{\(M \bigstep \numeral {n+1}\)}
  \AxiomC{\(N_2 \bigstep V\)}
  \BinaryInfC{\(\pcfifz(M,N_1,N_2) \bigstep V\)}
  \DisplayProof\qedhere
  \end{longtable}
\end{definition}

\begin{definition}[Value]
  A term is a \emph{value} if it is either a variable, a numeral or a
  \(\lambda\)-abstraction, i.e.\ it is of the form
  \(\lambdadot{\var x : \sigma}{N}\) for some term \(N\).

  Note that the values of type \(\pcfnat\) in the empty context are precisely
  the numerals.
\end{definition}

The reason that we use \(V\) in the big-step semantics is the following:

\begin{lemma}
  If \(M \bigstep V\), then \(V\) is a value.
\end{lemma}

Moreover, values do not reduce any further:

\begin{lemma}
  If \(V\) is a value, then
  \begin{enumerate}[(i)]
  \item there is no term \(N\) such that \(V \smallstep N\), and
  \item whenever \(V \bigstep N\), we have \(V = N\).
  \end{enumerate}
\end{lemma}

Thus, the big-step operational semantics reduces a term all the way to a value,
forgetting about the intermediary reductions.

Furthermore, the values computed by the big-step operational semantics are
unique:
\begin{lemma}[Big-step is deterministic]
  If \(M \bigstep V\) and \(M \bigstep W\), then \(V = W\).
\end{lemma}

%(Actually, the small-step operational semantics is also deterministic.)

The lemmas are proved by induction on the structure of derivations of
\(M \bigstep V\) and inspection of the small-step operational semantics.

Finally, we relate the small-step and big-step operational semantics:
%
We write \(\smallstep^\ast\) for the reflexive transitive closure of
\(\smallstep\). That is, we have \(M \smallstep^\ast N\) exactly if there is a
sequence \(M = M_0 \smallstep M_1 \smallstep \dots \smallstep M_{n-1} = N\). For
\(n = 0\), this reads \(M = N\), and for \(n = 1\), it reads \(M \smallstep N\).

\begin{exercise}\label{exer:small-and-big-step}
  Show that:
  \begin{enumerate}[(i)]
  \item\label{bigstep-implies-smallstep} if \(M \bigstep V\), then \(M \smallstep^\ast V\);
  \item\label{smallstep-gives-bigstep} if \(M \smallstep N\), then
    \(N \bigstep V\) implies \(M \bigstep V\) for all values \(V\);
  \item\label{smallstep-gives-bigstep'} if \(M \smallstep^\ast N\), then
    \(N \bigstep V\) implies \(M \bigstep V\) for all values \(V\).
  \end{enumerate}
  For~\ref{bigstep-implies-smallstep} and \ref{smallstep-gives-bigstep}, use
  induction on the structure of the derivations; for
  \ref{smallstep-gives-bigstep'} you can repeatedly
  apply~\ref{smallstep-gives-bigstep}.

  Conclude that \(M \bigstep V\) if and only if \(M \smallstep^\ast V\) for all
  terms \(M\) and values \(V\).
\end{exercise}

\section{List of exercises}
\begin{enumerate}
\item \cref{exer:small-step-addition}: On computing additions using the
  small-step operational semantics.
\item \cref{exer:pcf-add-mult}: On defining addition and multiplication in PCF.
\item \cref{exer:small-and-big-step}: On relating the small-step and big-step
  operational semantics.
\end{enumerate}




%%% Local Variables:
%%% mode: latexmk
%%% TeX-master: "../main"
%%% End:
