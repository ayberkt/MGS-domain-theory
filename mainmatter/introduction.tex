\chapter{Introduction}

\emph{Denotational semantics}, as pioneered by Scott and
Strachey~\cite{Scott1970,ScottStrachey1971}, aims to understand and reason about
computer programs by assigning mathematical meaning to the syntax of a
programming language.

While other choices for denotational semantics are possible, e.g.\ using
games~\cite{Abramsky1997,Hyland1997} or realizability~\cite{Longley1995}, these
notes employ a denotational semantics based on \emph{domain theory} to study the
functional programming language \emph{PCF}~\cite{Plotkin1977,Scott1993}.

The syntax (the types and terms) of PCF will be interpreted as certain kinds of
partially ordered sets and so-called continuous maps between them.
%
But a programming language is more than just its syntax: it should compute. The
\emph{operational semantics} of PCF specify its computational behaviour by
determining a reduction strategy for terms.

Following~\cite{Escardo2007}, we might summarise as:

\begin{displayquote}
  Operational semantics is about \emph{how} we compute.

  Denotational semantics is about \emph{what} we compute.
\end{displayquote}

The central theorems of \emph{soundness} and \emph{computational adequacy},
formulated and proved by Plotkin~\cite{Plotkin1977}, then tell us that a PCF
program computes to a value if and only if the denotational semantics of the
program and the value are equal.
%
Put differently, we might say that the operational and denotational semantics of
PCF are ``in sync''.

\section{Aims}

We hope that these notes provide a self-contained and accessible introduction to
domain-theoretic denotational semantics for graduate students in theoretical
computer science.

Domain theory~\cite{AbramskyJung1994,GierzEtAl2003} is a fruitful mathematical
subject with applications outside the semantics of programming languages, e.g.\
in algebra, higher-type computability~\cite{LongleyNormann2015} and
topology~\cite{GierzEtAl2003}.

For some, domain theory may be fairly abstract however, and for this reason we
have chosen to motivate the domain-theoretic definitions and constructions by
appealing to its application of modelling the programming language PCF.

In our investigations we are not only introduced to basic domain theory, but,
when proving computational adequacy, also to the fundamental technique of
logical relations.

Moreover, although no knowledge of category theory is required for these notes,
there are several exercises touching on category theory which hopefully provide
an illustration of abstract categorical concepts for those already familiar with
category theory, or, alternatively, might encourage those unfamiliar to study it.

More generally, we hope that this course will be taken up as an invitation to
explore the wonderful interplay between mathematics and computer science.
%
A striking example of this is Escard\'o's collection of ``seemingly impossible''
programs that perform fast exhaustive search on infinite datatypes
(see~\cite{Escardo2007b} and the references therein).
%
While the programs can be understood without knowledge of domain theory, their
conception and proofs of correctness do rely on domain theory and topology.

\section{References}

Our treatment is largely based on Streicher's account~\cite{Streicher2006},
although we have included several examples and proofs in an attempt to make our
notes more accessible.
%
Moreover, at times, we deviate from Streicher's treatment and follow Hart's Agda
formalisation~\cite{Hart2020} instead, e.g.\ we only consider well-typed terms
and no ``raw'' terms and
include~\cref{applicative-approximation-if-big-step-implication}.

Hart's Agda formalisation is the result of a final-year \emph{MSci} project
building on our constructive and predicative account of domain theory in the
topical univalent foundations (also known as homotopy type
theory)~\cite{deJong2022}. For accessibility reasons, these notes use a
classical set-theoretic foundation however.

\section{Further reading}

A natural resource for further reading is the aforementioned
textbook~\cite{Streicher2006}.
%
Additionally, one may wish to consult the
textbooks~\cite{Winskel1993,Gunther1992} or the lecture
notes~\cite{Plotkin1983,PittsWinskelFiore2012} for more on domain-theoretic denotational
semantics.
%
For more on general domain theory, we
recommend~\cite{AbramskyJung1994,GierzEtAl2003}.


%%% Local Variables:
%%% mode: latexmk
%%% TeX-master: "../main"
%%% End:
