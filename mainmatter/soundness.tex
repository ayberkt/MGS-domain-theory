\chapter{Soundness}\label{chap:soundness}

The soundness theorem expresses that the denotational semantics, in the form of
the Scott model of PCF, respects the operational semantics.

\begin{theorem}[Soundness]\label{soundness}
  For terms \(M\) and \(N\) of the same type in the same context, we have: if
  \(M \bigstep N\), then \(\densem{M} = \densem{N}\).
\end{theorem}
\begin{proof}
  By induction on the structure of the derivations of \(M \bigstep N\).
  %
  We work out two cases: the fixed point operator and application.

  Recall that the former is the rule
  \[
    \AxiomC{\(M(\pcffix_{\sigma} \, M) \bigstep V\)}
    \UnaryInfC{\({\pcffix_{\sigma} \, M} \bigstep V\)}
    \DisplayProof
  \]
  so that we may assume \(\densem{M(\pcffix_\sigma\,M)} = \densem{V}\) and we
  have to prove:
  \[
    \densem{\pcffix_{\sigma} \, M} = \densem{V}.
  \]
  But this holds, because if \(M\) is a term in context \(\Gamma\), then for
  every \(\gamma \in \densem{\Gamma}\), we have
  \begin{align*}
    \densem{\pcffix_{\sigma} \,M}(\gamma)
    &= \mu\pa*{\densem{M}(\gamma)}
    &\text{(by definition)} \\
    &= \pa*{\densem{M}(\gamma)}\pa*{\mu\pa*{\densem{M}(\gamma)}}
    &\text{(because \(\mu\) gives a fixed point)} \\
    &= \densem{M\pa*{\pcffix_{\sigma}\,M}}(\gamma)
    &\text{(by definition)} \\
    &= \densem{V}(\gamma) &\text{(by assumption)}.
  \end{align*}

  The big-step rule for application is
  \[
    \AxiomC{\(M \bigstep \lambdadot{\var x : \sigma}{E}\)}
    \AxiomC{\(E[N/x] \bigstep V\)}
    \BinaryInfC{\(M \, N \bigstep V\)}
    \DisplayProof
  \]
  so that we may assume
  \(\densem{M} = \densem{\lambdadot{\var x : \sigma}{E}}\)
  and
  \(\densem{E[N/x]} = \densem{V}\),
  and we have to prove:
  \[
    \densem{M \, N} = \densem{V}.
  \]
\end{proof}

%%% Local Variables:
%%% mode: latexmk
%%% TeX-master: "../main"
%%% End:
